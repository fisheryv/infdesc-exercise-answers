% !TeX root = ../../infdesc.tex
\section{Orders and lattices}


\begin{exercise}
Let $X$ be a set. The poset $(\mathcal{P}(X), \subseteq)$ has a least element and a greatest element; find both.
\end{exercise}

\begin{exercise}
Prove that the least element of $\mathbb{N}$ with respect to divisibility is $1$, and the greatest element is $0$.
\end{exercise}


\begin{exercise}
Let $X$ be a set, and let $U,V \in \mathcal{P}(X)$. Prove that the $\subseteq$-supremum of $\{ U, V \}$ is $U \cup V$, and the $\subseteq$-infimum of $\{ U, V \}$ is $U \cap V$.
\end{exercise}

\begin{exercise}
Let $a,b \in \mathbb{N}$. Show that $\mathrm{gcd}(a,b)$ is an infimum of $\{ a, b \}$ and that $\mathrm{lcm}(a,b)$ is a supremum of $\{ a, b \}$ with respect to divisibility.
\end{exercise}


\begin{exercise}[Properties of suprema and infima]
Let $(X, \preceq)$ be a lattice. Prove that, for all $x,y \in X$, we have:
\begin{enumerate}[(a)]
\item (\textbf{Idempotence}) $x \wedge x = x$ and $x \vee x = x$;
\item (\textbf{Commutativity}) $x \wedge y = y \wedge x$ and $x \vee y = y \vee x$;
\item (\textbf{Absorption}) $x \vee (x \wedge y) = x$ and $x \wedge (x \vee y) = x$.
\end{enumerate}
\end{exercise}


\begin{exercise}
\label{exNIsDistributiveLatticeUnderDivisibility}
Prove that $(\mathbb{N}, {\mid})$ is a distributive lattice.
\begin{backhint}
\hintref{exNIsDistributiveLatticeUnderDivisibility}
Use the characterisation of gcd and lcm in terms of prime factorisation.
\end{backhint}
\end{exercise}


\begin{exercise}
Let $(X, \preceq)$ be a distributive lattice with a greatest element and a least element, and let $x \in X$. Prove that, if a complement for $x$ exists, then it is unique; that is, prove that if $y,z \in X$ are complements for $X$, then $y=z$.
\hintlabel{exComplementIsUnique}{%
Use distributivity, together with the fact that $y = y \vee \bot$ and $y = y \wedge \top$.
}
\end{exercise}



\begin{exercise}
Prove part (b) of \Cref{thmDeMorgan}.
\hintlabel{exSecondPartOfDeMorgan}{%
This can be proved by swapping $\wedge$ with $\vee$ and $\top$ with $\bot$ in the proof of (a). But there is a shorter proof which uses the result of part (a) together with \Cref{propDoubleNegation}.
}
\end{exercise}