% !TeX root = ../../infdesc.tex
\section{Inductively defined sets}


\begin{exercise}
\label{exRulesForInductiveDefinitionOfWords}
Fix an alphabet $\Sigma$. Following \Cref{exWordsAsInductivelyDefinedSetPreliminary}, define rules that describe the set $\Sigma^*$ of words over $\Sigma$. How would your rules need to be changed if the empty word $\varepsilon$ were not allowed?
\end{exercise}



\begin{exercise}
Let $\Sigma = \{ a,b,c,d \}$. Draw a diagram to represent how the word $cbadbd \in \Sigma^*$ can be obtained by applying the rules you defined in \Cref{exRulesForInductiveDefinitionOfWords}.
\end{exercise}


\begin{exercise}
Let $\Sigma$ be an alphabet. Prove that $\Sigma^*$ is inductively defined by the rules $(~ \mid \varepsilon)$ and $\sigma_a = (w \mid wa)$ for $a \in \Sigma$.
\end{exercise}

\begin{exercise}
Prove that $\mathbb{N}$ is inductively defined by the rules $(~ \mid 0)$, $(~ \mid 1)$ and $(n \mid n+2)$.
\end{exercise}



\begin{exercise}
Let $R$ be a set of rules for an inductive definition and let $A$ be the set generated by $R$. Prove that if $R$ has no nullary rules, then $A$ is empty.
\end{exercise}

\begin{exercise}
Let $R$ be a set of rules for an inductive definition, and let $A$ be the set generated by $R$. Prove that if $R$ is countable, then $A$ is countable.
\end{exercise}

\begin{exercise}
\label{exSetGeneratedByRulesIsInductivelyDefined}
Let $R$ be a set of rules. Prove that the set $A$ generated by $R$ is inductively defined; for each rule $\sigma = (x_1,x_2,\dots,x_r \mid \sigma(x_1,x_2,\dots,x_r))$, the constructor $f_{\sigma} : A^r \to A$ is defined by
\[ f_{\sigma}(a_1,a_2,\dots,a_r) ~=~ \sigma(a_1,a_2,\dots,a_r) \]
where $\sigma(a_1,a_1,\dots,a_r)$ denotes the result of substituting $a_i$ for the variable $x_i$ for each $i \in [r]$.
\end{exercise}



\begin{exercise}
Let $\Sigma$ be an alphabet and let $\Sigma^*$ be the inductively defined set of words over $\Sigma$. Prove that for all $w \in \Sigma$, the rank $\mathrm{rk}(w)$ is equal to the length of $W$. [We regard the empty string $\varepsilon$ to have length $0$, despite the fact that the placeholder character $\varepsilon$ is used to denote it.]
\end{exercise}



\begin{exercise}
Let $P$ be a set of propositional variables. Prove by structural induction that the rank of a propositional formula $\varphi \in L(P)$ is equal to the number of logical operators in $\varphi$.
\end{exercise}



\begin{exercise}
Prove that the set $\mathbb{Z}^+$ of all positive integers is a quotient-inductive set given by the rules $(~ \mid 1)$ and $(x \mid x \cdot p)$ for each positive prime $p \in \mathbb{Z}$. Describe the corresponding inductively defined set $A$ and the quotient map $q : A \to \mathbb{Z}^+$ explicitly.
\end{exercise}

