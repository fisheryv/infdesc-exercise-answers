% !TeX root = ../../infdesc.tex
\section{Counting principles}
\secbegin{secCountingPrinciples}


\begin{exercise}
Fix $n \in \mathbb{N}$. Prove that $\binom{n}{0} = 1$, $\binom{n}{1} = n$ and $\binom{n}{n} = 1$.
\end{exercise}



\begin{exercise}
List all the permutations of the set $[4]$.
\end{exercise}


\begin{exercise}
Let $X$ and $Y$ be finite sets, and recall that $Y^X$ denotes the set of functions from $X$ to $Y$. Prove that $|Y^X|=|Y|^{|X|}$.
\hintlabel{exSizeOfFunctionSet}{%
Any function $f : X \to Y$ with finite domain can be specified by listing its values. For each $x \in X$, how many choices do you have for the value $f(x)$?
}
\end{exercise}

\begin{exercise}
Since September 2001, car number plates on the island of Great Britain have taken the form \texttt{XX\;NN\;XXX}, where each \texttt{X} can be any letter of the alphabet except for `I', `Q' or `Z', and \texttt{NN} is the last two digits of the year. [This is a slight simplification of what is really the case, but let's not concern ourselves with \textit{too} many details!] How many possible number plates are there? Number plates of vehicles registered in the region of Yorkshire begin with the letter `Y'. How many Yorkshire number plates can be issued in a given year?
\end{exercise}


\begin{exercise}
Count the number of injections $[3] \to [4]$.
\hintlabel{exNumberOfInjectionsThreeToFour}{%
The image (\Cref{defImage}) of an injection $[3] \to [4]$ must be a subset of $[4]$ of size three.
}
\end{exercise}


\begin{exercise}
Prove \Cref{thmAdditionPrinciple}. The proof follows the same pattern as that of \Cref{lemMultiplicationPrincipleIndependent}. Be careful to make sure you identify where you use the hypothesis that the sets $U_i$ are pairwise disjoint!
\end{exercise}



\begin{exercise}
Pick your favourite integer $n > 1000$. For this value of $n$, how many inhabited subsets of $[n]$ contain either only even or only odd numbers? (You need not evaluate exponents.)
\end{exercise}


\begin{exercise}
\label{exCityColourModified}
In \Cref{exCityColour}, how many ways could a committee be formed with a \textit{representative} number of people from each colour preference group? That is, the proportion of people on the committee which prefer any of the three colours should be equal to the corresponding proportion of the population of the city.
\end{exercise}


\begin{exercise}
Prove part (b) of \Cref{thmPigeonholePrinciple}.
\end{exercise}


\begin{exercise}
Six people are in a room. The atmosphere is tense, since each pair of people is either friends or enemies. There are no allegiances, so for example it is possible for a friend of a friend to be an enemy, or an enemy of a friend to be a friend, and so on. Prove that there is some set of three people that are either all each other's friends or all each other's enemies.
\end{exercise}


\begin{exercise}
Make the proof of \Cref{propBinomCoeffTwoColourBalls} more formal by defining a bijection between sets of the appropriate sizes.
\end{exercise}

\begin{exercise}
\label{exPascalIdentity}
Provide a combinatorial proof that if $n,k \in \mathbb{N}$ with $n \ge k$, then
\[ \binom{n+1}{k+1} = \binom{n}{k} + \binom{n}{k+1} \]

Deduce that the combinatorial definition of binomial coefficients (\Cref{defBinomialCoefficient}) is equivalent to the recursive definition (\Cref{defBinomialCoefficientRecursive}).
\begin{backhint}
\hintref{exPascalIdentity}
How many ways can you select $k+1$ animals from a set containing $n$ cats and one dog?
\end{backhint}
\end{exercise}


\begin{exercise}
\label{exCountingKTimesNChooseK}
Let $n,k \in \mathbb{N}$ with $k \le n+1$. Prove that
\[ k \binom{n}{k} = (n-k+1) \binom{n}{k-1} \]
\begin{backhint}
\hintref{exCountingKTimesNChooseK}
Find two procedures for counting the number of pairs $(U, u)$, such that $U \subseteq [n]$ is a $k$-element subset and $u \in U$. Equivalently, count the number of ways of forming a committee of size $k$ from a population of size $n$, and then appointing one member of the committee to be the chair.
\end{backhint}
\end{exercise}


\begin{exercise}
\label{exTrinomialCoefficients}
Given natural numbers $n,a,b,c$ with $a+b+c=n$, define the \textbf{trinomial coefficient} $\displaystyle \binom{n}{a,b,c}$\nindex{nChoosek3}{$\binom{n}{a,b,c}$}{trinomial coefficient} \index{trinomial coefficient} to be the number of ways of partitioning $[n]$ into three sets of sizes $a$, $b$ and $c$, respectively. That is, $\displaystyle \binom{n}{a,b,c}$ is the size of the set
\[ \left\{ (A,B,C)\ \middle|\ 
\begin{matrix} \begin{matrix} A \subseteq [n], & B \subseteq [n], & C \subseteq[n], \\ |A|=a, & |B|=b, & |C|=c, \end{matrix} \\ \text{and } A \cup B \cup C = [n] \end{matrix} \right\} \]
By considering trinomial coefficients, prove that if $a,b,c \in \mathbb{N}$, then $(a+b+c)!$ is divisible by $a! \cdot b! \cdot c!$.
\begin{backhint}
\hintref{exTrinomialCoefficients}
Find an expression for $(a+b+c)!$ in terms of $a!$, $b!$, $c!$ and $\binom{a+b+c}{a,b,c}$, following the pattern of \Cref{thmBinomAsFactorial}.
\end{backhint}
\end{exercise}
