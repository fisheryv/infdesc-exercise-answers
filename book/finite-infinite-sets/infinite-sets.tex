% !TeX root = ../../infdesc.tex
\section{Countable and uncountable sets}
\secbegin{secCountableUncountableSets}


\begin{exercise}
\label{exPositiveIntegersCountablyInfinite}
Prove that the set of all positive integers is countably infinite.
\end{exercise}

\begin{exercise}
\label{exEvenOddNaturalNumbersCountablyInfinite}
Prove that the set of all even natural numbers is countably infinite, and that the set of all odd natural numbers is countably infinite.
\end{exercise}


\begin{exercise}
Prove that the function $p : \mathbb{N} \times \mathbb{N} \to \mathbb{N}$ defined by $p(x,y) = 2^x(2y+1)-1$ is a bijection. Deduce that $\mathbb{N} \times \mathbb{N}$ is countable.
\hintlabel{exNCrossNIsCountable}{%
The fundamental theorem of arithmetic (\Cref{thmFTA}) implies that, for all $n \in \mathbb{N}$, we can express $n+1$ uniquely as a power of $2$ multiplied by an odd number. 
}
\end{exercise}

\begin{exercise}
Prove if $X$ and $Y$ are countably infinite sets, then $X \times Y$ is countably infinite.
\hintlabel{exProductOfTwoCountableSetsIsCountable}{%
Use \Cref{exCartesianProductOfBijections}, together with the definition of countably infinite sets, to construct a bijection $\mathbb{N} \times \mathbb{N} \to X \times Y$. Then apply \Cref{exNCrossNIsCountable} and \Cref{propCountablyInfiniteFromBijection}.
}
\end{exercise}



\begin{exercise}
\label{exCountabilityFromInjectionsAndSurjections}
Let $X$ be an inhabited set. The following are equivalent:
\begin{enumerate}[(i)]
\item \label{itmInjectionSurjectionCountableSets} $X$ is countable;
\item \label{itmInjectionIntoCountableSet} There exists an injection $f : X \to C$ for some countable set $C$;
\item \label{itmSurjectionFromCountableSet} There exists a surjection $f : C \to X$ for some countable set $C$.
\end{enumerate}
\hintlabel{exCountableFromInjSurjC}{%
For \ref{itmInjectionIntoCountableSet}$\Rightarrow$\ref{itmInjectionSurjectionCountableSets}, use the fact that the composite of two injections is injective. Likewise for \ref{itmSurjectionFromCountableSet}$\Rightarrow$\ref{itmInjectionSurjectionCountableSets}.
}
\end{exercise}



\begin{exercise} \label{exFiniteSubsetsCountableFixedSize}
Let $X$ be a countable set. Prove that $\binom{X}{k}$ is countable for each $k \in \mathbb{N}$.
\begin{backhint}
\hintref{exFiniteSubsetsCountableFixedSize}
Suppose $X=\mathbb{N}$. By \Cref{propFiniteProductOfCountableSetsIsCountable}, the set $\mathbb{N}^k$ is countable. By \Cref{thmCountableFromInjSurj}(c), it suffices to find an injection $\binom{\mathbb{N}}{k} \to \mathbb{N}^k$.
\end{backhint}
\end{exercise}



\begin{exercise}
Use Cantor's diagonal argument to prove that the set $\mathbb{N}^{\mathbb{N}}$ of all functions $\mathbb{N} \to \mathbb{N}$ is uncountable.
\end{exercise}


\begin{exercise}
Prove that if a set $X$ has an uncountable subset, then $X$ is uncountable.
\end{exercise}

\begin{exercise}
\label{exPowerSetFiniteOrUncountable}
Let $X$ be a set. Prove that $\mathcal{P}(X)$ is either finite or uncountable.
\begin{backhint}
\hintref{exPowerSetFiniteOrUncountable}
We have already proved this when $X$ is finite. When $X$ is countably infinite, use \Cref{thmCantorForN}. When $X$ is uncountably infinite, find an injection $X \to \mathcal{P}(X)$ and find a way to apply \Cref{exCountableFromInjSurjC}.
\end{backhint}
\end{exercise}


\begin{exercise}
Let $\Sigma = \{ 0,1,2,3,4,5,6,7,8,9,{\div},{-} \}$. Describe which words over $\Sigma$ immediately represent rational numbers (in the sense of treating the digit symbols as numbers and the symbols $\div$ and $-$ as arithmetic operations). Find some examples of elements of $\Sigma^*$ that do not represent rational numbers.
\hintlabel{exWordsRepresentingRationalNumbers}{%
How many `$\div$' symbols can a string from $\Sigma^*$ have if the string is to represent a rational number? Where in a word over $\Sigma$ can a $\div$ symbol appear?
}
\end{exercise}

\begin{exercise}
Let $\Sigma$ be an alphabet. Prove that if $\Sigma$ is countable, then $\Sigma^*$ is countable.
\hintlabel{exIfAlphabetCountableThenSetOfWordsCountable}{%
Prove by induction that $\Sigma^n$ is countable for all $n \in \mathbb{N}$, and then apply \Cref{thmCountableUnionOfCountableSetIsCountable}.
}
\end{exercise}



\begin{exercise}
\label{exFiniteDescriptionOfQ}
Give an explicit finite description of the rational numbers over a finite alphabet.
\end{exercise}

\begin{exercise}
\label{exFiniteDescriptionOfCofiniteSubsetsOfN}
Let $\mathcal{C}(\mathbb{N})$ be the set of all \textit{cofinite} subsets of $\mathbb{N}$; that is
\[ \mathcal{C}(\mathbb{N}) = \{ U \subseteq \mathbb{N} \mid \mathbb{N} \setminus U \text{ is finite} \} \]
Give an explicit finite description of the elements of $\mathcal{C}(\mathbb{N})$ over a finite alphabet.
\end{exercise}


\begin{exercise}
Prove that the set of all polynomials with rational coefficients is countable.
\end{exercise}

\begin{exercise}
Consider the following argument that claims to be a proof that $\mathcal{P}(\mathbb{N})$ is countable. (It isn't.)
\begin{quote}
Let $\Sigma = \left\{ \mathbb{N},x,U,V,\boxed{\mid},\boxed{\in},\boxed{\{},\boxed{\}} \right\}$---again, we have boxed some symbols to emphasise that they are elements of $\Sigma$.

Define $D : \mathcal{P}(\mathbb{N}) \to \Sigma^*$ by
\[ D(U) = \{ x \in \mathbb{N} \mid x \in U \} \quad \text{and} \quad D(V) = \{ x \in \mathbb{N} \mid x \in V \} \]
for all $U, V \subseteq \mathbb{N}$. Then $\Sigma$ is finite, and $D$ is injective since for all $U,V \subseteq \mathbb{N}$ we have
\[ D(U) = D(V) ~ \Rightarrow ~ \{ x \in \mathbb{N} \mid x \in U \} = \{ x \in \mathbb{N} \mid x \in V \} ~ \Rightarrow ~ U = V \]
Since $D$ is a finite description of the subsets of $\mathbb{N}$ over a finite alphabet, it follows from \Cref{thmCountableIffFiniteDescription} that $\mathcal{P}(\mathbb{N})$ is countable. \hfill $\Box$
\end{quote}
This proof cannot be valid, since we know that $\mathcal{P}(\mathbb{N})$ is uncountable. Where in the proof did we go wrong?
\hintlabel{exBadProofThatPowerSetOfNIsCountable}{%
Is $\Sigma$ \textit{really} finite? Is $D$ \textit{really} well-defined?
}
\end{exercise}

\begin{exercise}
Suppose we were to attempt to prove that $\mathcal{P}(\mathbb{N})$ is countable using the same argument as \Cref{exFiniteDescriptionOfFiniteSubsetsOfN}. Where does the proof fail?
\end{exercise}

\begin{exercise}
Prove that, for every countable set $X$, there is a finite description of the elements of $X$ over a \textit{finite} alphabet.
\hintlabel{exCountableIffFiniteDescriptionFiniteAlphabet}{%
Start by proving that there is a finite description of the elements of $\mathbb{N}^*$ over a finite alphabet $\Phi$. This defines an injection $\mathbb{N}^* \to \Phi^*$. Now given a countable alphabet $\Sigma$, use the injection $\Sigma \to \mathbb{N}$ given by \Cref{thmCountableFromInjSurj} to construct an injection $\Sigma^* \to \mathbb{N}^*$, and therefore an injection $\Sigma^* \to \Phi^*$. Finally, use this to turn finite descriptions over $\Sigma$ into finite descriptions over $\Phi$.
}
\end{exercise}