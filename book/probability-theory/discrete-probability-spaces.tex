% !TeX root = ../../infdesc.tex
\section{Discrete probability spaces}
\secbegin{secDiscreteProbabilitySpaces}



\begin{exercise}
Let $(\Omega,\mathbb{P})$ be a probability space. Prove that
\[ \mathbb{P}(\Omega \setminus A) = 1-\mathbb{P}(A) \]
for all events $A$.
\end{exercise}



\begin{exercise}
A fair six-sided die is rolled twice. Define a probability space $(\Omega, \mathbb{P})$ that models this situation.
\end{exercise}

\begin{exercise}
\label{exProbabilityOfSubset}
Let $(\Omega,\mathbb{P})$ be a probability space and let $A,B$ be events with $A \subseteq B$. Prove that $\mathbb{P}(A) \le \mathbb{P}(B)$.
\end{exercise}


\begin{exercise}
\label{exFormalOrAndNot}
Let $(\Omega, \mathbb{P})$ be a probability space, and let $p(\omega),q(\omega)$ be logical formulae with free variable $\omega$ ranging over $\Omega$. Let
\[ A = \{ \omega \in \Omega \mid p(\omega) \} \quad \text{and} \quad B = \{ \omega \in \Omega \mid q(\omega) \} \]
Prove that
\begin{itemize}
\item $\{ \omega \in \Omega \mid p(\omega) \wedge q(\omega) \} = A \cap B$;
\item $\{ \omega \in \Omega \mid p(\omega) \vee q(\omega) \} = A \cup B$;
\item $\{ \omega \in \Omega \mid \neg p(\omega) \} = A^c$.
\end{itemize}
For reference, in \Cref{exInformalOrAndNot}, we had $\Omega = [6] \times [6]$ and we defined $p(a,b)$ to be `$a+b$ is even' and $q(a,b)$ to be `$a+b \ge 7$'.
\end{exercise}



\begin{exercise}
Prove parts (ii) and (iii) of \Cref{propIndicatorFunctionSetOperations}.
\end{exercise}

\begin{exercise}
\label{exProbabilityWithIndicatorFunction}
Let $(\Omega, \mathbb{P})$ be a discrete probability space, and for each $\omega \in \Omega$ let $p_{\omega} = \mathbb{P}(\{\omega\})$. Prove that, for each event $A$, we have
\[ \mathbb{P}(A) = \sum_{\omega \in \Omega} p_{\omega}i_A(\omega) \]
\end{exercise}


\begin{exercise}
Let $(\Omega,\mathbb{P})$ be a probability space. Under what conditions is an event $A$ independent from itself?
\end{exercise}



\begin{exercise}
A fair six-sided die is rolled three times. What is the probability that the sum of the die rolls is less than or equal to $12$, given that each die roll shows a power of $2$?
\end{exercise}

\begin{exercise}
\label{exConditionalProbabilityOfIntersection}
Let $(\Omega,\mathbb{P})$ be a probability space and let $A,B$ be events with $\mathbb{P}(B)>0$. Prove that
\[ \mathbb{P}(A \mid B) = \mathbb{P}(A \cap B \mid B) \]
\end{exercise}

\begin{exercise}
Let $(\Omega,\mathbb{P})$ be a probability space and let $A,B$ be events such that $\mathbb{P}(B)>0$. Prove that $\mathbb{P}(A \mid B) = \mathbb{P}(A)$ if and only if $A$ and $B$ are independent.
\end{exercise}


\begin{exercise}
In \Cref{exBayesCarCompany}, find the probabilities that the car was an Allegheny and that the car was a Monongahela.
\end{exercise}