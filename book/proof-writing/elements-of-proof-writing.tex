% !TeX root = ../../infdesc.tex
\section{Elements of proof-writing}
\label{secElementsOfProofWriting}


\begin{exercise}
Consider the following proof that the product of two odd integers is odd.
\begin{quote}
$a,b \in \mathbb{Z}$ both odd $\Rightarrow$ $a=2k+1$, $b=2\ell+1$, $k,\ell \in \mathbb{Z}$
\[ ab = (2k+1)(2\ell+1) = 4k\ell+2k+2\ell+1 = 2(2k\ell+k+\ell)+1\]
$2k\ell + k + \ell \in \mathbb{Z}$ $\therefore$ $ab$ odd
\end{quote}
Rewrite the proof so that it is easier to read it as prose. Read it aloud, and transcribe what you read.
\end{exercise}



\begin{exercise}
Consider the following notation-heavy proof that $X \cap (Y \cup Z) \subseteq (X \cap Y) \cup (X \cap Z)$ for all sets $X$, $Y$ and $Z$.
\begin{quote}
Let $X$, $Y$ and $Z$ be sets and let $a \in X \cap (Y \cup Z)$. Then
\[ a \in X \wedge a \in Y \cup Z ~ \Rightarrow ~ a \in X \wedge (a \in Y \vee a \in Z) \]
\begin{itemize}
\item \textbf{Case 1:} $a \in Y \Rightarrow (a \in X \wedge a \in Y) \Rightarrow a \in X \cap Y \Rightarrow a \in (X \cap Y) \cup (X \cap Z)$.
\item \textbf{Case 2:} $a \in Z \Rightarrow (a \in X \wedge a \in Z) \Rightarrow a \in X \cap Y \Rightarrow a \in (X \cap Y) \cup (X \cap Z)$.
\end{itemize}
$\therefore \forall a,~ a \in X \cap (Y \cup Z) \Rightarrow a \in (X \cap Y) \cup (X \cap Z)$

$\therefore$ $X \cap (Y \cup Z) \subseteq (X \cap Y) \cup (X \cap Z)$.
\end{quote}
Read the proof aloud and transcribe what you said. Then rewrite the proof with a more appropriate balance of notation and text.
\end{exercise}

