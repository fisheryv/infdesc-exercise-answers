% !TeX root = ../../infdesc.tex
\section{Weak induction}
\secbegin{secWeakInduction}

\begin{exercise}
\label{exSumOfPowersOf2}
Prove by induction that $\displaystyle \sum_{k=0}^n 2^k = 2^{n+1} - 1$ for all $n \in \mathbb{N}$.
\end{exercise}


\begin{exercise}
Write down the statement $p(n)$ that \Cref{exHorseInductionFail} attempted to prove for all $n \ge 1$. Convince yourself that the proof of the base case is correct, then write down---with quantifiers---exactly the proposition that the induction step is meant to prove. Explain why the argument in the induction step failed to prove this proposition.
\end{exercise}


\begin{exercise}
\label{exFormulaForGeometricProgression}
Let $a,r \in \mathbb{R}$ with $r \ne 1$. Then
\[ \sum_{k=0}^n ar^k = \frac{a(1-r^{n+1})}{1-r} \]
for all $n \in \mathbb{N}$.
\end{exercise}


\begin{exercise}
Let $n \in \mathbb{N}$ and let $\{ X_k \mid 1 \le k \le n \}$ be a family of sets. Prove by induction on $n$ that there is a bijection $\displaystyle \prod_{k=1}^{n+1} X_k \to \left( \prod_{k=1}^n X_k \right) \times X_n$.
\hintlabel{exProductOfSuccNSets}{%
To define the bijection, think about what the elements of the two sets look like: The elements of $\prod_{k=1}^{n+1} X_k$ look like $(a_1, a_2, \dots, a_n, a_{n+1})$, where $a_k \in X_k$ for each $1 \le k \le n+1$. On the other hand, the elements of $\left( \prod_{k=1}^n X_k \right) \times X_{n+1}$ look like $((a_1, a_2, \dots, a_n), a_{n+1})$.
}
\end{exercise}


\begin{exercise}
Prove that $\dbinom{n}{k} = \dbinom{n}{n-k}$ for all $n,k \in \mathbb{N}$ with $k \le n$.
\end{exercise}


\begin{exercise}
Use the binomial theorem to prove that
\[ \sum_{i=0}^n (-1)^i \binom{n}{i} = 0 \]
\hintlabel{exAlternatingSumOfBinomialCoefficientsIsZero}{%
Note that $(-1)^i \dbinom{n}{i} = \dbinom{n}{i} \cdot (-1)^i \cdot 1^{n-i}$.
}
\end{exercise}
