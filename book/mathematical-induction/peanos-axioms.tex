% !TeX root = ../../infdesc.tex
\section{Peano's axioms}
\secbegin{secPeanosAxioms}

The purpose of this section is to forget everything we think we know about the natural numbers, and reconstruct our former knowledge (and more!)\ using the following fundamental property:

\begin{center}
\textit{Every natural number can be obtained in a unique way by\\
starting from zero and adding one some finite number of times.}
\end{center}

This is slightly imprecise---it is not clear what is meant by `adding one some finite number of times', for example. Worse still, we are going to define what `finite' means in terms of natural numbers in \Cref{secFiniteSets}, so we'd better not refer to finiteness in our definition of natural numbers!

The following definition captures precisely the properties that we need in order to characterise the idea of $\mathbb{N}$ that we have in our minds. To begin with, $\mathbb{N}$ should be a set. Whatever the elements of this set $\mathbb{N}$ actually \textit{are}, we will think about them as being natural numbers. One of the elements, in particular, should play the role of the natural number $0$---this will be the \textit{zero element} $z \in \mathbb{N}$; and there should be a notion of `adding one'---this will be the \textit{successor function} $s : \mathbb{N} \to \mathbb{N}$. Thus given an element $n \in \mathbb{N}$, though of as a natural number, we think about the element $s(n)$ as the natural number `$n+1$'. Note that this is strictly for the purposes of intuition: we will define `$+$' and `$1$' in terms of $z$ and $s$, not vice versa.

\begin{definition}
\label{defNotionOfNaturalNumbers}
\index{natural numbers!notion of}
\index{Peano's axioms}
A \textbf{notion of natural numbers} is a set $\mathbb{N}$, together with an element $z \in \mathbb{N}$, called a \textbf{zero element}, and a function $s : \mathbb{N} \to \mathbb{N}$ called a \textbf{successor function}, satisfying the following properties:
\begin{enumerate}[(i)]
\item $z \not\in s[\mathbb{N}]$; that is, $z \ne s(n)$ for any $n \in \mathbb{N}$.
\item $s$ is injective; that is, for all $m,n \in \mathbb{N}$, if $s(m) = s(n)$, then $m=n$.
\item $\mathbb{N}$ is generated by $z$ and $s$; that is, for all sets $X$, if $z \in X$ and for all $n \in \mathbb{N}$ we have $n \in X \Rightarrow s(n) \in X$, then $\mathbb{N} \subseteq X$.
\end{enumerate}
The properties (i), (ii) and (iii) are called \textbf{Peano's axioms}.
\end{definition}

Note that \Cref{defNotionOfNaturalNumbers} does not specify what $\mathbb{N}$, $z$ and $s$ actually are; it just specifies the properties that they must satisfy. It turns out that it doesn't really matter what notion of natural numbers we use, since any two notions are essentially the same. We will not worry about the specifics here---that task is left to \Cref{secConstructions}: a particular notion of natural numbers is defined in \Cref{cnsNaturalNumbersVonNeumann}, and the fact that all notions of natural numbers are `essentially the same' is made precise and proved in \Cref{thmNNNUnique}.

We can define all the concepts involving natural numbers that we are familiar with, and prove all the properties that we take for granted, just from the element $z \in \mathbb{N}$ and the successor function $s : \mathbb{N} \to \mathbb{N}$.

For instance, we define `$0$' to mean $z$, define `$1$' to mean $s(z)$, define `$2$' to mean $s(s(z))$, and so on. Thus `$12$' is defined to mean
\[ s(s(s(s(s(s(s(s(s(s(s(s(z)))))))))))) \]

From now on, then, let's write $0$ instead of $z$ for the zero element of $\mathbb{N}$. It would be nice if we could write `$n+1$' instead of $s(n)$, but we must first define what `$+$' means. In order to do this, we need a way of defining expressions involving natural numbers; this is what the \textit{recursion theorem} allows us to do.

\begin{theorem}[Recursion theorem]
\label{thmRecursion}
\index{recursion!recursion theorem}
Let $X$ be a set. For all $a \in X$ and all $h : \mathbb{N} \times X \to X$, there is a unique function $f : \mathbb{N} \to X$ such that $f(0) = a$ and $f(s(n)) = h(n, f(n))$ for all $n \in \mathbb{N}$.
\end{theorem}

\begin{cproof}
Let $a \in X$ and $h : \mathbb{N} \times X \to X$. We prove existence and uniqueness of $f$ separately.

\begin{itemize}
\item Define $f : \mathbb{N} \to X$ by specifying $f(0) = a$ and $f(s(n)) = h(n, f(n))$. Since $h$ is a function and $s$ is injective, existence and uniqueness of $x \in X$ such that $f(n) = x$ is guaranteed, provided that $f(n)$ is defined, so we need only verify totality.

So let $D = \{ n \in \mathbb{N} \mid f(n) \text{ is defined} \}$. Then:

\begin{itemize}
\item $0 \in D$, since $f(0)$ is defined to be equal to $a$.
\item Let $n \in \mathbb{N}$ and suppose $n \in D$. Then $f(n)$ is defined and $f(s(n)) = h(n, f(n))$, so that $f(s(n))$ is defined, and hence $s(n) \in D$.
\end{itemize}

%% BEGIN EXTRACT (xtrConclusionExample) %%
By condition (iii) of \Cref{defNotionOfNaturalNumbers}, we have $\mathbb{N} \subseteq D$, so that $f(n)$ is defined for all $n \in \mathbb{N}$, as required.
%% END EXTRACT %%

\item To see that $f$ is unique, suppose $g : \mathbb{N} \to X$ were another function such that $g(0) = a$ and $g(s(n)) = h(n, g(n))$ for all $n \in \mathbb{N}$.

To see that $f = g$, let $E = \{ n \in \mathbb{N} \mid f(n) = g(n) \}$. Then
\begin{itemize}
\item $0 \in E$, since $f(0) = a = g(0)$.
\item Let $n \in \mathbb{N}$ and suppose that $n \in E$. Then $f(n) = g(n)$, and so
\[ f(s(n)) = h(n,f(n)) = h(n,g(n)) = g(s(n)) \]
and so $s(n) \in E$.
\end{itemize}
Again, condition (iii) of \Cref{defNotionOfNaturalNumbers} is satisfied, so that $\mathbb{N} \subseteq E$. It follows that $f(n) = g(n)$ for all $n \in \mathbb{N}$, and so $f=g$.
\end{itemize}

Thus we have established the existence and uniqueness of a function $f : \mathbb{N} \to X$ such that $f(0) = a$ and $f(s(n)) = h(n, f(n))$ for all $n \in \mathbb{N}$.
\end{cproof}

The recursion theorem allows us to define expressions involving natural numbers \textit{by recursion}; this is \Cref{strDefinitionByRecursion}.

\begin{strategy}[Definition by recursion]
\label{strDefinitionByRecursion}
\index{recursion!definition by recursion}
In order to specify a function $f : \mathbb{N} \to X$, it suffices to define $f(0)$ and, for given $n \in \mathbb{N}$, assume that $f(n)$ has been defined, and define $f(s(n))$ in terms of $n$ and $f(n)$.
\end{strategy}

\begin{example}
We can use recursion to define addition on the natural numbers as follows.

For fixed $m \in \mathbb{N}$, we can define a function $\mathrm{add}_m : \mathbb{N} \to \mathbb{N}$ by recursion by:
\[ \mathrm{add}_m(0) = m \quad \text{and} \quad \mathrm{add}_m(s(n)) = s(\mathrm{add}_m(n)) \text{ for all } n \in \mathbb{N} \]
In more familiar notation, for $m,n \in \mathbb{N}$, define the expression `$m+n$' to mean $\mathrm{add}_m(n)$. Another way of expressing the recursive definition of $\mathrm{add}_m(n)$ is to say that, for each $m \in \mathbb{N}$, we are defining $m+n$ by recursion on $n$ as follows:
\[ m+0 = m \quad \text{and} \quad m+s(n) = s(m+n) \text{ for all } n \in \mathbb{N} \]
\end{example}

We can use the recursive definition of addition to prove familiar equations between numbers. The following proposition is a proof that $2+2=4$. This may seem silly, but notice that the expression `$2+2=4$' is actually shorthand for the following:
\[ \mathrm{add}_{s(s(0))} (s(s(0))) = s(s(s(s(0)))) \]
We must therefore be careful to apply the definitions in its proof.

\begin{proposition}
\label{propTwoPlusTwoEqualsFour}
$2+2=4$
\end{proposition}

\begin{cproof}
We use the recursive definition of addition.
\begin{align*}
2 + 2 &= 2 + s(1) && \text{since $2=s(1)$} \\
&= s(2+1) && \text{by definition of $+$} \\
&= s(2+s(0)) && \text{since $1=s(0)$} \\
&= s(s(2+0)) && \text{by definition of $+$} \\
&= s(s(2)) && \text{by definition of $+$} \\
&= s(3) && \text{since $3=s(2)$} \\
&= 4 && \text{since $4=s(3)$}
\end{align*}
as required.
\end{cproof}

The following result allows us to drop the notation `$s(n)$' and just write `$n+1$' instead.

\begin{proposition}
\label{propSuccessorIsPlusOne}
For all $n \in \mathbb{N}$, we have $s(n) = n+1$.
\end{proposition}

\begin{cproof}
Let $n \in \mathbb{N}$. Then by the recursive definition of addition we have
\[ n+1 = n+s(0) = s(n+0) = s(n) \]
as required.
\end{cproof}

In light of \Cref{propSuccessorIsPlusOne}, we will now abandon the notation $s(n)$, and write $n+1$ instead.

We can define the arithmetic operations of multiplication and exponentiation by recursion, too.

\begin{example}
Fix $m \in \mathbb{N}$. Define $m \cdot n$ for all $n \in \mathbb{N}$ by recursion on $n$ as follows:
\[ m \cdot 0 = 0 \quad \text{and} \quad m \cdot (n+1) = (m \cdot n) + m \text{ for all } n \in \mathbb{N} \]
Formally, what we have done is define a function $\mathrm{mult}_m : \mathbb{N} \to \mathbb{N}$ recursively by $\mathrm{mult}_m(z)=z$ and $\mathrm{mult}_m(s(n)) = \mathrm{add}_{\mathrm{mult}_m(n)}(m)$ for all $n \in \mathbb{N}$. But the definition we provided is easier to understand.
\end{example}

\begin{proposition}
\label{propTwoTimesTwoEqualsFour}
$2 \cdot 2 = 4$
\end{proposition}

\begin{cproof}
We use the recursive definitions of addition and recursion.
\begin{align*}
2 \cdot 2 &= 2 \cdot (1+1) && \text{since $2=1+1$} \\
&= (2 \cdot 1) + 2 && \text{by definition of $\cdot$} \\
&= (2 \cdot (0+1)) + 2 && \text{since $1 = 0 + 1$} \\
&= ((2 \cdot 0) + 2) + 2 && \text{by definition of $\cdot$} \\
&= (0+2) + 2 && \text{by definition of $\cdot$} \\
&= (0+(1+1)) + 2 && \text{since $2=1+1$} \\
&= ((0+1)+1) + 2 && \text{by definition of $+$} \\
&= (1+1) + 2 && \text{since $1=0+1$} \\
&= 2+2 && \text{since $2=1+1$} \\
&= 4 && \text{by \Cref{propTwoPlusTwoEqualsFour}}
\end{align*}
as required.
\end{cproof}

\begin{exercise}
Given $m \in \mathbb{N}$, define $m^n$ for all $n \in \mathbb{N}$ by recursion on $n$, and prove that $2^2 = 4$ using the recursive definitions of exponentiation, multiplication and addition.
\end{exercise}

We could spend the rest of our lives doing long computations involving recursively defined arithmetic operations, so at this point we will stop, and return to taking for granted the facts that we know about arithmetic operations.

There are, however, a few more notions that we need to define by recursion so that we can use them in our proofs later.

\begin{definition}
\label{defSumOfRealNumbers}
Given $n \in \mathbb{N}$, the \textbf{sum} of $n$ real numbers $a_1, a_2, \dots, a_n$ is the real number $\sum_{k=1}^n a_k$ defined by recursion on $n \in \mathbb{N}$ by
\[ \sum_{k=1}^0 a_k = 0 \quad \text{and} \quad \sum_{k=1}^{n+1} a_k = \left( \sum_{k=1}^n a_k \right) + a_{n+1} \text{ for all } n \in \mathbb{N} \]
\end{definition}

\begin{definition}
\label{defProductOfRealNumbers}
Given $n \in \mathbb{N}$, the \textbf{product} of $n$ real numbers $a_1, a_2, \dots, a_n$ is the real number $\prod_{k=1}^n a_k$ defined by recursion on $n \in \mathbb{N}$ by
\[ \prod_{k=1}^0 a_k = 1 \quad \text{and} \quad \prod_{k=1}^{n+1} a_k = \left( \prod_{k=1}^n a_k \right) \cdot a_{n+1} \text{ for all } n \in \mathbb{N} \]
\end{definition}

\begin{example}
Let $x_i=i^2$ for each $i \in \mathbb{N}$. Then
\[ \sum_{i=1}^5 x_i = 1 + 4 + 9 + 16 + 25 = 55 \]
and
\[ \prod_{i=1}^5 x_i = 1 \cdot 4 \cdot 9 \cdot 16 \cdot 25 = 14400 \]
\end{example}

\begin{exercise}
Let $x_1, x_2 \in \mathbb{R}$. Working strictly from the definitions of indexed sum and indexed product, prove that
\[ \sum_{i=1}^2 x_i = x_1 + x_2 \quad \text{and} \quad \prod_{i=1}^2 x_i = x_1 \cdot x_2 \]
\end{exercise}

\subsection*{Binomials and factorials}

\begin{definition}[to be redefined in \Cref{defFactorial}]
\label{defFactorialRecursive}
\index{factorial}
Let $n \in \mathbb{N}$. The \textbf{factorial} of $n$, written $n!$\nindex{nfactorial}{$n"!$}{factorial}, is defined recursively by
\[ 0! = 1 \quad \text{and} \quad (n+1)! = (n+1) \cdot n! \text{ for all } n \ge 0 \]
\end{definition}

Put another way, we have
\[ n! = \prod_{i=1}^n i \]
for all $n \in \mathbb{N}$---recall \Cref{defProductOfRealNumbers} to see why these definitions are really just two ways of wording the same thing.

\begin{definition}[to be redefined in \Cref{defBinomialCoefficient}]
\label{defBinomialCoefficientRecursive}
\index{binomial coefficient}
\nindex{nchoosek}{$\binom{n}{k}$}{binomial coefficient}
Let $n,k \in \mathbb{N}$. The \textbf{binomial coefficient} $\binom{n}{k}$ \inlatex{binom\{n\}\{k\}}\lindexmmc{binom}{$\binom{n}{k}$} (read `$n$ choose $k$') is defined by recursion on $n$ \textit{and} on $k$ by
\[ \binom{n}{0}=1, \quad \binom{0}{k+1} = 0, \quad \binom{n+1}{k+1} = \binom{n}{k} + \binom{n}{k+1} \]
\end{definition}

This definition gives rise to an algorithm for computing binomial coefficients: they fit into a diagram known as \textbf{Pascal's triangle}\index{Pascal's triangle}, with each binomial coefficient computed as the sum of the two lying above it (with zeroes omitted):

\begin{center}\begin{tabular}{ccc}
$\binom{0}{0}$ && $1$ \\
$\binom{1}{0}$ \quad $\binom{1}{1}$ && $1$ \quad $1$ \\
$\binom{2}{0}$ \quad $\binom{2}{1}$ \quad $\binom{2}{2}$ & = & $1$ \quad $2$ \quad $1$ \\
$\binom{3}{0}$ \quad $\binom{3}{1}$ \quad $\binom{3}{2}$ \quad $\binom{3}{3}$ && $1$ \quad $3$ \quad $3$ \quad $1$ \\
$\binom{4}{0}$ \quad $\binom{4}{1}$ \quad $\binom{4}{2}$ \quad $\binom{4}{3}$ \quad $\binom{4}{4}$ && $1$ \quad $4$ \quad $6$ \quad $4$ \quad $1$ \\
$\binom{5}{0}$ \quad $\binom{5}{1}$ \quad $\binom{5}{2}$ \quad $\binom{5}{3}$ \quad $\binom{5}{4}$ \quad $\binom{5}{5}$ && $1$ \quad $5$ \quad $10$ \quad $10$ \quad $5$ \quad $1$ \\
$\vdots$ \qquad $\vdots$ \qquad $\vdots$ \qquad $\vdots$ \qquad $\vdots$ && $\vdots$ \qquad $\vdots$ \qquad $\vdots$ \qquad $\vdots$
\end{tabular}\end{center}

\begin{exercise}
Write down the next two rows of Pascal's triangle.
\end{exercise}