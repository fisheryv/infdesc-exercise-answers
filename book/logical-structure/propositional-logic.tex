% !TeX root = ../../infdesc.tex
\section{Propositional logic}
\secbegin{secPropositionalLogic}



\begin{exercise}
\label{exJohnMathematicianPittsburgh}
Express the proposition `John is a mathematician who lives in Pittsburgh' in the form $p \wedge q$, for propositions $p$ and $q$.
\end{exercise}


\begin{exercise}
Write a proof tree whose conclusion is the propositional formula $(p \wedge q) \wedge (r \wedge s)$, where $p,q,r,s$ are propositional variables. Express `$2$ is an even prime number and $3$ is an odd prime number' in the form $(p \wedge q) \wedge (r \wedge s)$, for appropriate propositions $p$, $q$, $r$ and $s$, and describe how your proof tree suggests what a proof might look like.
\end{exercise}


\begin{exercise}
Let $n$ be an integer. Prove that $n^2$ leaves a remainder of $0$, $1$ or $4$ when divided by $5$.
\hintlabel{exSquareRemainderModuloFive}{f
Mimic the proof of \Cref{propRemainderOfSquaresModulo3}.
}
\end{exercise}

\begin{exercise}
Let $a,b \in \mathbb{R}$ and suppose $a^2-4b \ne 0$. Let $\alpha$ and $\beta$ be the (distinct) roots of the polynomial $x^2+ax+b$. Prove that there is a real number $c$ such that either $\alpha-\beta = c$ or $\alpha - \beta = ci$.
\end{exercise}


\begin{exercise}
Let $p(x)$ be a polynomial over $\mathbb{C}$. Prove that if $\alpha$ is a root of $p(x)$, and $a \in \mathbb{C}$, then $\alpha$ is a root of $(x-a)p(x)$.
\end{exercise}


\begin{exercise}
\label{exQuadraticFormulaConverse}
Complete the proof of the quadratic formula. That is, for fixed $a,b \in \mathbb{C}$, prove that if
\[
\alpha = \frac{-a+\sqrt{a^2-4b}}{2} \quad \text{or} \quad \alpha =\frac{-a-\sqrt{a^2-4b}}{2}
\]
then $\alpha$ is a root of the polynomial $x^2+ax+b$.
\end{exercise}


\begin{exercise}
Prove that a natural number $n$ is divisible by $3$ if and only if the sum of its base-$10$ digits is divisible by $3$.
\hintlabel{exSumOfDigitsDivisibleByThree}{%
Suppose $n = d_r \cdot 10^r + \cdots + d_1 \cdot 10 + d_0$ and let $s = d_r + \cdots + d_1 + d_0$. Start by proving that $3 \mid n-s$.
}
\end{exercise}


\begin{exercise}
\label{exNegationAndReciprocalOfIrrationalNumbers}
Let $x \in \mathbb{R}$. Prove by contradiction that if $x$ is irrational then $-x$ and $\frac{1}{x}$ are irrational.
\end{exercise}

\begin{exercise}
\label{exNoLeastPositiveReal}
Prove by contradiction that there is no least positive real number. That is, prove that there is not a positive real number $a$ such that $a \le b$ for all positive real numbers $b$.
\end{exercise}


\begin{exercise}
Reflect on the proof of \Cref{propIfProductEvenThenSomeFactorEven}. Where in the proof did we use the law of excluded middle? Where in the proof did we use proof by contradiction? What was the contradiction in this case? Prove \Cref{propIfProductEvenThenSomeFactorEven} twice more, once using contradiction and not using the law of excluded middle, and once using the law of excluded middle and not using contradiction.
\end{exercise}

\begin{exercise}
Let $a$ and $b$ be irrational numbers. By considering the number $\sqrt{2}^{\sqrt{2}}$, prove that it is possible that $a^b$ be rational.
\hintlabel{exIrrationalExpIrrationalCanBeRational}{%
Use the law of excluded middle according to whether the proposition `$\sqrt{2}^{\sqrt{2}}$ is rational' is true or false.}
\end{exercise}


