% !TeX root = ../../infdesc.tex
The goal of this chapter is to develop a methodical way of breaking up a proposition into smaller components and seeing how these components fit together---this is called the \textit{logical structure} of a proposition. The logical structure of a proposition is very informative: it tells us what we need to do in order to prove it, what we need to write in order to communicate our proof, and how to explore the consequences of the proposition after it has been proved.

\begin{center} \vspace{-15pt}
\fitwidth{%
\begin{tikzcd}[row sep={30pt}, column sep={20pt}, ampersand replacement=\&]
\&
\text{\begin{minipage}{90pt}\centering logical structure of a proposition\end{minipage}}
\arrow[dl]
\arrow[d]
\arrow[dr]
\&
\\
\text{\begin{minipage}{120pt}\centering strategies for proving\\ the proposition\end{minipage}}
\&
\text{\begin{minipage}{100pt}\centering structure and wording of the proof \end{minipage}}
\&
\text{\begin{minipage}{100pt}\centering consequences of\\ the proposition\end{minipage}}
\end{tikzcd}%
}
\end{center}

\Cref{secPropositionalLogic,secVariablesQuantifiers} are dedicated to developing a system of \textit{symbolic logic} for reasoning about propositions. We will be able to represent a proposition using a string of variables and symbols, and this expression will guide how we can prove the proposition and explore its consequences. In \Cref{secLogicalEquivalence} we will develop techniques for manipulating these logical expressions algebraically---this, in turn, will yield new proof techniques (some have fancy names like `proof by contraposition', but some do not).

Exploring how the logical structure of a proposition informs the structure and wording of its proof is the content of \Cref{secVocabulary}.