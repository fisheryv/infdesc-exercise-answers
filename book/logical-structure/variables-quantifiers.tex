% !TeX root = ../../infdesc.tex
\section{Variables and quantifiers}
\secbegin{secVariablesQuantifiers}


\begin{exercise}
\label{exEnglishToLogicalFormulae}
Find logical formulae that represent each of the following English statements.
\begin{enumerate}[(a)]
\item There is an integer that is divisible by every integer.
\item There is no greatest odd integer.
\item Between any two distinct rational numbers is a third distinct rational number.
\item If an integer has a rational square root, then that root is an integer.
\end{enumerate}
\end{exercise}

\begin{exercise}
\label{exLogicalFormulaeToEnglish}
Find statements in plain English, involving as few variables as possible, that are represented by each of the following logical formulae. (The domains of discourse of the free variables are indicated in each case.)
\begin{enumerate}[(a)]
\item $\exists q \in \mathbb{Z},\, a = qb$ --- free variables $a, b \in \mathbb{Z}$
\item $\exists a \in \mathbb{Z},\, \exists b \in \mathbb{Z},\, (b \ne 0 \wedge bx = a)$ --- free variable $x \in \mathbb{R}$
\item $\forall d \in \mathbb{N},\, [(\exists q \in \mathbb{Z},\, n=qd) \Rightarrow (d = 1 \vee d = n)]$ --- free variable $n \in \mathbb{N}$
\item $\forall a \in \mathbb{R},\, [a > 0 \Rightarrow \exists b \in \mathbb{R},\, (b > 0 \wedge a < b)]$ --- no free variables
\end{enumerate}
\end{exercise}


\begin{exercise}
\label{exEveryIntegerIsRational}
Prove that every integer is rational.
\end{exercise}

\begin{exercise}
Prove that every linear polynomial over $\mathbb{Q}$ has a rational root.
\end{exercise}

\begin{exercise}
Prove that, for all real numbers $x$ and $y$, if $x$ is irrational, then $x+y$ and $x-y$ are not both rational.
\hintlabel{exIrrationalPlusMinusRealNotBothRational}{%
Consider the sum of $x+y$ and $x-y$.
}
\end{exercise}



\begin{exercise}
Prove that there is a real number which is irrational but whose square is rational.
\end{exercise}

\begin{exercise}
Prove that there is an integer which is divisible by zero.
\hintlabel{exZeroDividesSomeInteger}{%
Look carefully at the definition of divisibility (\Cref{defDivisionPreliminary}).
}
\end{exercise}



\begin{exercise}
\label{exExamplesOfUniqueExistentialQuantifier}
For each of the propositions, write it out as a logical formula involving the $\exists !$ quantifier and then prove it, using the structure of the logical formula as a guide.
\begin{enumerate}[(a)]
\item For each real number $a$, the equation $x^2+2ax+a^2=0$ has exactly one real solution $x$.
\item There is a unique real number $a$ for which the equation $x^2+a^2=0$ has a real solution $x$.
\item There is a unique natural number with exactly one positive divisor.
\end{enumerate}
\end{exercise}


\begin{exercise}
Let $p(x,y)$ be the statement `$x + y$ is even'.
\begin{itemize}
\item Prove that $\forall x \in \mathbb{Z},\, \exists y \in \mathbb{Z},\, p(x,y)$ is true.
\item Prove that $\exists y \in \mathbb{Z},\, \forall x \in \mathbb{Z},\, p(x,y)$ is false.
\end{itemize}
\end{exercise}
