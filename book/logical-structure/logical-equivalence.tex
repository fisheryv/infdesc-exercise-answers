% !TeX root = ../../infdesc.tex
\section{Logical equivalence}
\secbegin{secLogicalEquivalence}


\begin{exercise}
\label{exPAndQImpliesRIffPImpliesRAndQImpliesR}
Let $p$, $q$ and $r$ be propositional variables. Prove that the propositional formula $(p \vee q) \Rightarrow r$ is logically equivalent to $(p \Rightarrow r) \wedge (q \Rightarrow r)$.
\end{exercise}


\begin{exercise}
Use the definitions of $\wedge$, $\vee$ and $\Rightarrow$ to justify the truth tables in \Cref{exNegationConjunctionDisjunctionImplicationTruthTable}.
\hintlabel{exJustifyBasicTruthTables}{Note that you may need to use the law of excluded middle (\Cref{axLEM}) and the principle of explosion (\Cref{axPrincipleOfExplosion}).}
\end{exercise}

\begin{exercise}
\label{exImplicationInTermsOfDisjunctionWithTruthTables}
Use a truth table to prove that $p \Rightarrow q \equiv (\neg p) \vee q$.
\end{exercise}

\begin{exercise}
\label{exPAndQImpliesRIffPImpliesRAndQImpliesRWithTruthTables}
Let $p$, $q$ and $r$ be propositional variables. Use a truth table to prove that the propositional formula $(p \vee q) \Rightarrow r$ is logically equivalent to $(p \Rightarrow r) \wedge (q \Rightarrow r)$.
\end{exercise}


\begin{exercise}
Use the law of contraposition to prove that $p \Leftrightarrow q \equiv (p \Rightarrow q) \wedge ((\neg p) \Rightarrow (\neg q))$, and use the proof technique that this equivalence suggests to prove that an integer is even if and only if its square is even.
\hintlabel{exIntegerEvenIffSquareEven}{%
Express this statement as $\forall n \in \mathbb{Z},\, (n \text{ is even}) \Leftrightarrow (n^2 \text{ is even})$, and note that the negation of `$x$ is even' is `$x$ is odd'.
}
\end{exercise}


\begin{exercise}
Prove that $p \vee q \equiv (\neg p) \Rightarrow q$. Use this logical equivalence to suggest a new strategy for proving propositions of the form $p \vee q$, and use this strategy to prove that if two integers sum to an even number, then either both integers are even or both are odd.
\end{exercise}


\begin{exercise}
Prove \Cref{thmDeMorganLogicalOperators}(b) thrice: once using the definition of logical equivalence directly (like we did in \Cref{exConjunctionDistributesOverDisjunction,exImplicationInTermsOfDisjunction} and \Cref{exPAndQImpliesRIffPImpliesRAndQImpliesR}), once using a truth table, and once using part (a) together with the law of double negation.
\end{exercise}


\begin{exercise}
\label{exNegationOfImplication}
Prove that $\neg (p \Rightarrow q) \equiv p \wedge (\neg q)$ twice, once using a truth table, and once using \Cref{exImplicationInTermsOfDisjunctionWithTruthTables} together with de Morgan's laws and the law of double negation.
\end{exercise}



\begin{exercise}
Prove by counterexample that not every rational number can be expressed as $\dfrac{a}{b}$ where $a \in \mathbb{Z}$ is even and $b \in \mathbb{Z}$ is odd.
\hintlabel{exNotEveryRationalIsEvenDividedByOdd}{%
Find a rational number all of whose representations as a ratio of two integers have an even denominator.
}
\end{exercise}


\begin{exercise}
Determine which of the following logical formulae are maximally negated.
\begin{enumerate}[(a)]
\item $\forall x \in X,\, (\neg p(x)) \Rightarrow \forall y \in X, \neg (r(x,y) \wedge s(x,y))$;
\item $\forall x \in X,\, (\neg p(x)) \Rightarrow \forall y \in X, (\neg r(x,y)) \vee (\neg s(x,y))$;
\item $\forall x \in \mathbb{R},\, [x > 1 \Rightarrow (\exists y \in \mathbb{R},\, [x < y \wedge \neg (x^2 \le y)])]$;
\item $\neg \exists x \in \mathbb{R},\, [x > 1 \wedge (\forall y \in \mathbb{R},\, [x < y \Rightarrow x^2 \le y])]$.
\end{enumerate}
\end{exercise}


\begin{exercise}
Find a maximally negated propositional formula that is logically equivalent to $\neg (p \Leftrightarrow q)$.
\hintlabel{exMaximallyNegateBiconditional}{%
Start by expressing $\Leftrightarrow$ in terms of $\Rightarrow$ and $\wedge$, as in \Cref{defBiconditional}.
}
\end{exercise}

\begin{exercise}
Maximally negate the following logical formula, then prove that it is true or prove that it is false.
\[ \exists x \in \mathbb{R},\, [x > 1 \wedge (\forall y \in \mathbb{R},\, [x < y \Rightarrow x^2 \le y])]\]
\end{exercise}


\begin{exercise}
\label{exTautologies}
Prove that each of the following is a tautology:
\begin{enumerate}[(a)]
\item $[(p \Rightarrow q) \wedge (q \Rightarrow r)] \Rightarrow (p \Rightarrow r)$;
\item $[p \Rightarrow (q \Rightarrow r)] \Rightarrow [(p \Rightarrow q) \Rightarrow (p \Rightarrow r)]$;
\item $\exists y \in Y,\, \forall x \in X,\, p(x,y) \Rightarrow \forall x \in X,\, \exists y \in Y,\, p(x,y)$;
\item $[\neg (p \wedge q)] \Leftrightarrow [(\neg p) \vee (\neg q)]$;
\item $(\neg \forall x \in X,\, p(x)) \Leftrightarrow (\exists x \in X,\, \neg p(x))$.
\end{enumerate}
For each, try to interpret what it means, and how it might be useful in a proof.
\end{exercise}

