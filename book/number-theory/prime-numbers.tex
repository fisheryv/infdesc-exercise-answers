% !TeX root = ../../infdesc.tex
\section{Prime numbers}
\secbegin{secPrimes}


\begin{exercise}
Using \Cref{defPrime}, prove that $3$ and $5$ are prime, and that $4$ is not prime.
\end{exercise}



\begin{exercise}
Let $p \in \mathbb{Z}$ be a positive prime and let $0 < k < p$. Show that $p \mid \binom{p}{k}$.
\hintlabel{exPrimeDivBinomCoeff}{%
Use the factorial formula for binomial coefficients (\Cref{thmBinomAsFactorialByInduction}).
}
\end{exercise}


\begin{exercise}
Let $p \in \mathbb{Z}$. Prove that if $p$ is ring theoretically prime, then $p$ is irreducible.
\hintlabel{exPrimeImpliesIrred}{%
Assume $p=mn$ for some $m,n \in \mathbb{Z}$. Prove that $m$ or $n$ is a unit.
}
\end{exercise}


\begin{exercise}
\label{exNumbersAsProductsOfPrimes}
Express the following numbers as products of primes:
\[ 16 \qquad {-240} \qquad 5050 \qquad 111111 \qquad {-123456789} \]
\end{exercise}


\begin{exercise}
Write out the canonical prime factorisations of the numbers from \Cref{exNumbersAsProductsOfPrimes}, which were:
\[ 16 \qquad {-240} \qquad 5050 \qquad 111111 \qquad {-123456789} \]
\end{exercise}


\begin{exercise}
\label{exGCDfromFTA}
Let $p_1,p_2,\dots,p_r$ be distinct primes, and let $k_i,\ell_i \in \mathbb{N}$ for all $1 \le i \le r$. Define
\[ m = p_1^{k_1} \times p_2^{k_2} \times \cdots \times p_r^{k_r} \quad \text{and} \quad n = p_1^{\ell_1} \times p_2^{\ell_2} \times \cdots \times p_r^{\ell_r} \]
Prove that
\[ \mathrm{gcd}(m,n) = p_1^{u_1} \times p_2^{u_2} \times \cdots \times p_r^{u_r} \]
where $u_i = \mathrm{min} \{ k_i, \ell_i \}$ for all $1 \le i \le r$.
\end{exercise}


\begin{exercise}
\label{exConstructPrimeNotInList}
Let $P$ be an inhabited finite set of positive prime numbers and let $m$ be the product of all the elements of $P$. That is, for some $n \ge 1$ let
\[ P = \{ p_1, \dots, p_n \} \quad \text{and} \quad m = p_1 \times \cdots \times p_n \]
where each $p_k \in P$ is a positive prime. Using the fundamental theorem of arithmetic, show that $m+1$ has a positive prime divisor which is not an element of $P$.
\end{exercise}


\begin{exercise}
\label{exPrimesUsingFactorials}
Let $n \in \mathbb{Z}$ with $n>2$. Prove that the set $\{ k \in \mathbb{Z} \mid n<k<n! \}$ contains a prime number.
\begin{backhint}
\hintref{exPrimesUsingFactorials}
What are the prime factors of $n!-1$?
\end{backhint}
\end{exercise}