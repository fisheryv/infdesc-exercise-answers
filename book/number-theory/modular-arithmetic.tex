% !TeX root = ../../infdesc.tex
\section{Modular arithmetic}
\secbegin{secModularArithmetic}


\begin{exercise}
For which integers $x$ does the congruence $5x+1 \equiv x+8 \bmod 3$ hold? Characterise such integers $x$ in terms of their remainder when divided by $3$.
\end{exercise}



\begin{exercise}
Fix a modulus $n$. Is it true that
\[ a \equiv b \bmod n \quad \Rightarrow \quad a^k \equiv b^k \bmod n \] for all $a,b \in \mathbb{Z}$ and $k \in \mathbb{N}$? If so, prove it; if not, provide a counterexample.
\end{exercise}

\begin{exercise}
Fix a modulus $n$. Is it true that
\[ k \equiv \ell \bmod n \quad \Rightarrow \quad a^k \equiv a^{\ell} \bmod n \]
for all $k,\ell \in \mathbb{N}$ and $a \in \mathbb{Z}$? If so, prove it; if not, provide a counterexample.
\end{exercise}

\begin{exercise}
Fix a modulus $n$. Is it true that
\[ qa \equiv qb \bmod n \quad \Rightarrow \quad a \equiv b \bmod n \]
for all $a,b,q \in \mathbb{Z}$ with $q \not \equiv 0 \bmod n$? If so, prove it; if not, provide a counterexample.
\end{exercise}


\begin{exercise}
For $n=7,8,9,10,11,12$, either find a multiplicative inverse for $6$ modulo $n$, or show that no multiplicative inverse exists. Can you spot a pattern?
\end{exercise}



\begin{exercise}
Let $n$ be a modulus and let $a \in \mathbb{Z}$. Suppose that $u$ is a multiplicative inverse for $a$ modulo $n$. Prove that, for all $k \in \mathbb{Z}$, $u+kn$ is a multiplicative inverse for $a$ modulo $n$.
\end{exercise}


\begin{exercise}
Find all integers $x$ such that $25x-4 \equiv 4x+3 \bmod 13$.
\end{exercise}



\begin{exercise}
\label{exPowerModN}
Let $n$ be a modulus and let $a \in \mathbb{Z}$ with $a \perp n$. Prove that there exists $k \ge 1$ such that $a^k \equiv 1 \bmod n$.
\begin{backhint}
\hintref{exPowerModN}
Consider the list $a^0, a^1, a^2, \dots$. Since there are only finitely many remainders modulo $n$, we must have $a^i \equiv a^j \bmod n$ for some $0 \le i < j$.
\end{backhint}
\end{exercise}


\begin{exercise}
\label{RemainderOfThreeExpBigRemThirteen}
Find the remainder of $3^{244886}$ when divided by $13$.
\begin{backhint}
\hintref{RemainderOfThreeExpBigRemThirteen}
First find the remainder of $244886$ when divided by $12$.
\end{backhint}
\end{exercise}


\begin{exercise}
\label{exTotientMultiplyByPrime}
Let $n \in \mathbb{Z}$ and let $p > 0$ be prime. Prove that if $p \mid n$, then $\varphi(pn) = p \cdot \varphi(n)$. Deduce that $\varphi(p^k) = p^k-p^{k-1}$ for all prime $p>0$ and all $k \ge 1$.
\begin{backhint}
\hintref{exTotientMultiplyByPrime}
Find a bijection $[p] \times C_n \to C_{pn}$, where $C_n = \{ k \in [|n|] \mid k \perp n \}$. You will need to use the techniques of \Cref{secFiniteSets} in your proof.
\end{backhint}
\end{exercise}

% \begin{exercise}
% \label{exTotientOfProductOfTwoPrimes}
% Let $p$ and $q$ be distinct positive primes. Prove that $\varphi(pq)=(p-1)(q-1)$.
% \begin{backhint}
% \hintref{exTotientOfProductOfTwoPrimes}
% Start by proving that $k \in [pq]$ is \textit{not} coprime to $pq$ if and only if $p \mid k$ or $q \mid k$. You will need to use the techniques of \Cref{secFiniteSets} in your proof.
% \end{backhint}
% \end{exercise}

\begin{exercise}
Let $n \in \mathbb{Z}$ and let $p>0$ be prime. Prove that if $p \nmid n$, then $\varphi(pn)=(p-1)\varphi(n)$.
\hintlabel{exTotientMultiplyByPrimeTwo}{%
Start by proving that $k \in [pn]$ is \textit{not} coprime to $pn$ if and only if either $p \mid k$ or $k$ is not coprime to $n$. You will need to use the techniques of \Cref{secFiniteSets} in your proof.
}
\end{exercise}



\begin{exercise}
Prove that $\varphi(100)=40$, this time using the inclusion--exclusion principle.
\end{exercise}



\begin{exercise}
Use Euler's theorem to prove that the last two digits of $3^{79}$ are `$67$'.
\hintlabel{exLastTwoDigitsOfPower}{%
Recall that $\varphi(100) = 40$---this was \Cref{exTotientOfOneHundred}.
}
\end{exercise}


\begin{exercise} \label{exCompositeDividesFactorial}
Let $n \in \mathbb{Z}$ be composite. Prove that if $n>4$, then $n \mid (n-1)!$.
\end{exercise}

\begin{exercise} \label{exSquareCongruentToOneModPrimeUnit}
Let $p$ be a positive prime and let $a \in \mathbb{Z}$. Prove that, if $a^2 \equiv 1 \bmod p$, then $a \equiv 1 \bmod p$ or $a \equiv {-1} \bmod p$. 
\begin{backhint}
\hintref{exSquareCongruentToOneModPrimeUnit}
You need to use the fact that $p$ is prime at some point in your proof.
\end{backhint}
\end{exercise}



\begin{exercise}
\label{exPrimeFactorialCongruentToMinusOne}
Let $p$ be a positive prime. Prove that $(p-1)! \equiv -1 \bmod p$.
\begin{backhint}
\hintref{exPrimeFactorialCongruentToMinusOne}
Pair as many elements of $[p-1]$ as you can into multiplicative inverse pairs modulo $p$.
\end{backhint}
\end{exercise}

\begin{exercise}
Let $p$ be an odd positive prime. Prove that
\[ \left[ \left( \frac{p-1}{2} \right)! \right]^2 \equiv (-1)^{\frac{p+1}{2}} \bmod p \]
\end{exercise}


\begin{exercise}
Find all integer solutions $x$ to the system of congruences:
\[ \begin{cases}
x \equiv {-1} \bmod 4 \\
x \equiv 1 \bmod 9 \\
x \equiv 5 \bmod{11}
\end{cases} \]
Express your solution in the form $x \equiv a \bmod n$ for suitable $n > 0$ and $0 \le a < n$.
\end{exercise}

\begin{exercise}
\label{exCRTExistence}
Let $m,n$ be coprime moduli and let $a,b \in \mathbb{Z}$. Let $u,v \in \mathbb{Z}$ be such that
\[ mu \equiv 1 \bmod n \quad \text{and} \quad nv \equiv 1 \bmod m \]
In terms of $a,b,m,n,u,v$, find an integer $x$ such that
\[ x \equiv a \bmod m \quad \text{and} \quad x \equiv b \bmod n \]
\end{exercise}

\begin{exercise}
\label{exCRTUniqueness}
Let $m,n$ be coprime moduli and let $x,y \in \mathbb{Z}$. Prove that if $x \equiv y \bmod m$ and $x \equiv y \bmod n$, then $x \equiv y \bmod{mn}$.
\end{exercise}


\begin{exercise}
\label{exCRTAlgorithm}
Verify that the algorithm outlined above is correct. Use it to compute the solutions to the system of congruences
\[ x \equiv 3 \bmod{12} \quad \text{and} \quad x \equiv 15 \bmod{20} \]
\end{exercise}

\begin{oexercise}
\label{exGeneralisedCRT}
Generalise the Chinese remainder theorem to systems of arbitrarily (finitely) many congruences. That is, given $r \in \mathbb{N}$, find precisely the conditions on moduli $n_1,n_2,\dots,n_r$ and integers $a_1,a_2,\dots,a_r$ such that an integer solution exists to the congruences
\[ x \equiv a_1 \bmod{n_1}, \quad x \equiv a_2 \bmod{n_2}, \qquad \cdots \qquad x_r \equiv a_r \bmod{n_r} \]
Find an explicit formula for such a value of $x$, and find a suitable modulus $n$ in terms of $n_1,n_2,\dots,n_r$ such that any two solutions to the system of congruences are congruent modulo $n$.
\begin{backhint}
\hintref{exGeneralisedCRT}
This generalisation will be tricky! You may need to generalise the definitions and results about greatest common divisors and least common multiples that we have seen so far, including B\'{e}zout's lemma. You might want to try proving this first in the case that $n_i \perp n_j$ for all $i \ne j$.
\end{backhint}
\end{oexercise}

\begin{exercise}
\label{exGapsBetweenPrimesLarge}
Prove that gaps between consecutive primes can be made arbitrarily large. That is, prove that for all $n \in \mathbb{N}$, there exists an integer $a$ such that the numbers
\[ a,\ a+1,\ a+2,\ \dots,\ a+n \]
are all composite.
\begin{backhint}
\hintref{exGapsBetweenPrimesLarge}
Observe that if $a,k \in \mathbb{Z}$ and $k \mid a$, then $k \mid a+k$.
\end{backhint}
\end{exercise}


\begin{exercise}
Let $n \in \mathbb{N}$. Prove that $n$ is divisible by $5$ if and only if the final digit in the decimal expansion of $n$ is $5$ or $0$.

More generally, fix $k \ge 1$ and let $m$ be the number whose decimal expansion is given by the last $k$ digits of that of $n$. Prove that $n$ is divisible by $5^k$ if and only if $m$ is divisible by $5^k$. For example, we have
\[ 125 \mid 9\;550\;828\;230\;495\;875 \quad \Leftrightarrow \quad 125 \mid 875 \]
\end{exercise}

\begin{exercise}
Let $n \in \mathbb{N}$. Prove that $n$ is divisible by $11$ if and only if the \textit{alternating sum} of the digits of $n$ is divisible by $11$. That is, prove that if the decimal expansion of $n$ is $d_rd_{r-2} \cdots d_0$, then
\[ 11 \mid n \quad \Leftrightarrow \quad 11 \mid d_0 - d_1 + d_2 - \cdots + (-1)^rd_r \]
\end{exercise}

\begin{exercise}
Let $n \in \mathbb{N}$. Find a method for testing if $n$ is divisible by $7$ based on the decimal expansion of $n$.
\end{exercise}


\begin{exercise}
Prove that the RSA algorithm is correct. Specifically, prove:
\begin{enumerate}[(a)]
\item If $n=pq$, for distinct positive primes $p$ and $q$, then $\varphi(n) = (p-1)(q-1)$;
\item Given $1<e<\varphi(n)$ with $e \perp \varphi(n)$, there exists $d \in \mathbb{Z}$ with $de \equiv 1 \bmod \varphi(n)$.
\item Given $M,K \in \mathbb{Z}$ with $K \equiv M^e \bmod n$, it is indeed the case that $K^d \equiv M \bmod n$.
\end{enumerate}
\end{exercise}
