% !TeX root = ../../infdesc.tex
\section{Equivalence relations and partitions}
\secbegin{secEquivalenceRelationsPartitions}


\begin{exercise}
\label{exEquivalenceRelationFromFunction}
Given a function $f : X \to Y$, define a relation $\sim_f$ on $X$ by
\[ a \sim_f b \quad \Leftrightarrow \quad f(a) = f(b) \]
for all $a,b \in X$. Prove that $\sim_f$ is an equivalence relation on $X$.
\end{exercise}



\begin{exercise}
\label{exBijectionIsEquivalenceRelation}
Let $\mathcal{S}$ be some set whose elements are all sets. (For example, we could take $\mathcal{S} = \mathcal{P}(X)$ for some fixed set $X$.) Define a relation $\cong$ \inlatex{cong}\lindexmmc{cong}{$\cong$} on $\mathcal{S}$ by letting $U \cong V$ if and only if there exists a bijection $f : U \to V$, for all $U,V \in \mathcal{S}$. Prove that $\cong$ is an equivalence relation on $\mathcal{S}$.
\end{exercise}



\begin{exercise}
Let $n \in \mathbb{Z}$. Prove that if $n \ne 0$, then $a \equiv b \bmod n$ if and only if $a$ and $b$ have the same remainder when divided by $n$.
\end{exercise}

\begin{exercise}
Let $a,b \in \mathbb{Z}$. When is it true that $a \equiv b \bmod 0$? When is it true that $a \equiv b \bmod 1$?
\end{exercise}


\begin{exercise}
Let $\approx$ be the relation of congruence modulo $10$ on $\mathbb{N}$. Describe the equivalence classes, and give an explicit expression of the quotient $\mathbb{N}/{\approx}$ in list notation.
\end{exercise}


\begin{exercise}
Let $f : X \to Y$ be a function. Prove that $f$ is injective if and only if each $\sim_f$-equivalence class has a unique element, where $\sim_f$ is the equivalence relation defined in \Cref{exEquivalenceRelationFromFunction}.
\end{exercise}


\begin{exercise}
\label{exCongruenceClassesCorrespondWithRemainders}
Let $n \in \mathbb{Z}$ with $n \ne 0$. Prove that the function
\[ i : \{ 0,~ 1,~ \dots,~ |n|-1 \} \to \mathbb{Z}/n\mathbb{Z} \]
defined by $i(r) = [r]_n$ for all $0 \le r < |n|$ is a bijection.
\end{exercise}


\begin{exercise}
\label{exPreimagesFormPartition}
Let $f : X \to Y$ be a surjection, and define a collection $\mathcal{F}$ of subsets of $X$ by
\[ \mathcal{F} = \{ f^{-1}[\{b\}] \mid b \in Y \} \]
That is, $\mathcal{F}$ is the set of subsets of $X$ given by the preimages of individual elements of $Y$ under $f$. Prove that $\mathcal{F}$ is a partition of $X$. Where in your proof do you use surjectivity of $f$?
\end{exercise}

\begin{exercise}
\label{exConditionsForPartition}
Let $X$ be a set and let $\mathcal{U} = \{ U_i \mid i \in I \}$ be a family of inhabited subsets of $X$. Prove that $\mathcal{U}$ is a partition of $X$ if and only if for eeach $a \in X$, there is a unique set $U_i \in \mathcal{U}$ with $a \in U_i$.
\end{exercise}

\begin{exercise}
\label{exQuotientIsPartition}
If $\sim$ be an equivalence relation on $X$, then $X/{\sim}$ is a partition $X$.
\end{exercise}


\begin{exercise}
\label{exBijectionOfQuotientsAndClassesInducesBijectionOfSets}
Let $X$ and $Y$ be sets, let $\sim$ be an equivalence relation on $X$ and let $\approx$ be an equivalence relation on $Y$. Assume that there is a bijection $p : X/{\sim} \to Y/{\approx}$, and for each equivalence class $E \in X/{\sim}$ there is a bijection $h_E : E \to p(E)$. Use \Cref{lemBijectionBetweenPartitionAndComponentsInducesBijectionOfSets} to prove that there is a bijection $h : X \to Y$.
\end{exercise}


\begin{exercise}
Let $n \in \mathbb{Z}$ with $n \ne 0$. Describe the quotient function $q_n : \mathbb{Z} \to \mathbb{Z}/n\mathbb{Z}$ in terms of remainders.
\end{exercise}

\begin{exercise}
\label{exQuotientFunctionIsSurjective}
Let $\sim$ be an equivalence relation on a set $X$. Prove that the quotient function $q_{\sim} : X \to X/{\sim}$ is surjective.
\end{exercise}


\begin{exercise}
Give an explicit description of the dashed arrow in the above diagram. That is, describe the correspondence between partitions of a set $X$ and surjections whose domain is $X$.
\hintlabel{exPartitionsSurjections}{%
Given a partition $\mathcal{U}$ of a set $X$, find a surjection $q : X \to \mathcal{U}$. Then prove that, for every surjection $p : X \to A$, there is a unique partition $\mathcal{U}_p$ of $X$ and a unique bijection $f : \mathcal{U}_p \to A$ such that, for all $U \in \mathcal{U}_p$, we have $p(x) = f(U)$ for all $x \in U$. The structure of the proof will be similar to that of \Cref{thmEquivalenceRelationsSurjections}.
}
\end{exercise}
