% !TeX root = ../../infdesc.tex
\section{Injections and surjections}
\secbegin{secInjectionsSurjections}


\begin{exercise}
Let $f : X \to Y$ and $g : Y \to Z$ be functions. Prove that if $g \circ f$ is injective, then $f$ is injective.
\end{exercise}

\begin{exercise}
Write out what it means to say a function $f : X \to Y$ is \textit{not} injective, and say how you would prove that a given function is not injective. Give an example of a function which is not injective, and use your proof technique to write a proof that it is not injective.
\end{exercise}

\begin{exercise}
For each of the following functions, determine whether it is injective or not injective.
\begin{itemize} 
\item $f : \mathbb{N} \to \mathbb{Z}$, defined by $f(n)=n^2$ for all $n \in \mathbb{N}$.
\item $g : \mathbb{Z} \to \mathbb{N}$, defined by $g(n)=n^2$ for all $n \in \mathbb{Z}$.
\item $h : \mathbb{N} \times \mathbb{N} \times \mathbb{N} \to \mathbb{N}$, defined by $h(x,y,z) = 2^x \cdot 3^y \cdot 5^z$ for all $x,y,z \in \mathbb{N}$.
\end{itemize}
\end{exercise}

\begin{exercise}
\label{exLinearPolynomialIsInjective}
Let $a,b \in \mathbb{R}$ with $b \ne 0$, and define $f : \mathbb{R} \to \mathbb{R}$ by $f(t) = a+bt$ for all $t \in \mathbb{R}$. Prove that $f$ is injective.
\end{exercise}


\begin{exercise}
For each of the following pairs of sets $(X,Y)$, determine whether the function $f : X \to Y$ defined by $f(x)=2x+1$ is surjective.
\begin{enumerate}[(a)]
\item $X = \mathbb{Z}$ and $Y = \{ x \in \mathbb{Z} \mid x \text{ is odd} \}$;
\item $X = \mathbb{Z}$ and $Y = \mathbb{Z}$;
\item $X = \mathbb{Q}$ and $Y = \mathbb{Q}$;
\item $X = \mathbb{R}$ and $Y = \mathbb{R}$.
\end{enumerate}
\end{exercise}

\begin{exercise}
Let $f : X \to Y$ be a function. Find a subset $V \subseteq Y$ and a surjection $g : X \to V$ agreeing with $f$ (that is, such that $g(x)=f(x)$ for all $x \in X$).
\hintlabel{exSurjectiveCorestriction}{%
Recall \Cref{defImage}.
}
\end{exercise}

\begin{exercise}
Let $f : X \to Y$ be a function. Prove that $f$ is surjective if and only if $Y=f[X]$
\end{exercise}

\begin{exercise}
\label{exEpiMonoFactorisation}
Let $f : X \to Y$ be a function. Prove that there is a set $Z$ and functions
\[ p : X \to Z \quad \text{and} \quad i : Z \to Y \]
such that $p$ is surjective, $i$ is injective, and $f = i \circ p$.
\begin{backhint}
\hintref{exEpiMonoFactorisation}
If $Z$ were a subset of $Y$, then we could easily define an injection $i : Z \to Y$ by $i(z)=z$ for all $z \in Z$. Are there any subsets of $Y$ that are associated with a function whose codomain is $Y$?
\end{backhint}
\end{exercise}

\begin{exercise}
Let $f : X \to \mathcal{P}(X)$ be a function. By considering the set $B = \{ x \in X \mid x \not\in f(x) \}$, prove that $f$ is not surjective.
\hintlabel{exRussellSubset}{%
Note that $B \in \mathcal{P}(X)$. Prove that there does not exist $a \in X$ such that $B = f(a)$.
}
\end{exercise}


\begin{exercise}
\label{exIdentityBijection}
Let $X$ be a set. Prove that the identity function $\mathrm{id}_X : X \to X$ is a bijection.
\end{exercise}

\begin{exercise}
\label{exCompositeOfBijectionsIsBijection}
Let $f : X \to Y$ and $g : Y \to Z$ be bijections. Prove that $g \circ f$ is a bijection.
\end{exercise}


\begin{exercise}
Let $f : X \to Y$ be a function. Prove that $f$ is injective if and only if
\[ \forall y \in f[X],\, \exists ! x \in X,\, y=f(x) \]
\end{exercise}



\begin{exercise}
\label{exEncodingPairs}
Define a function $e : \mathbb{N} \times \mathbb{N} \to \mathbb{N}$ by $e(m,n) = 2^m \cdot 3^n$. Prove that $e$ is injective. We can think of $e$ as encoding \textit{pairs} of natural numbers as single natural numbers---for example, the pair $(4,1)$ is encoded as $2^4 \cdot 3^1 = 48$. For each of the following natural numbers $k$, find the pairs of natural numbers encoded by $e$ as $k$:
\[ 1 \qquad 24 \qquad 7776 \qquad 59049 \qquad 396718580736 \]
\end{exercise}


\begin{exercise}
\label{exIfHasLeftInverseThenInjective}
Let $f : X \to Y$ be a function. Prove that if $f$ has a left inverse, then $f$ is injective.
\end{exercise}


\begin{exercise}
Let $f : X \to Y$ be a function with left inverse $g : Y \to X$. Prove that $g$ is a surjection.
\hintlabel{exLeftInversesAreSurjective}{%
This can be proved in a single sentence; if you find yourself writing a long proof, then there is an easier way.
}
\end{exercise}


\begin{exercise}
Let $f : X \to Y$ be a function. Prove that if $f$ has a right inverse, then $f$ is surjective.
\hintlabel{exIfHasRightInverseThenSurjective}{%
The proof is almost identical to \Cref{exLeftInversesAreSurjective}.
}
\end{exercise}


\begin{exercise}
\label{exFindTwoSidedInverses}
The following functions have two-sided inverses. For each, find its inverse and prove that it is indeed an inverse.
\begin{enumerate}[(a)]
\item $f : \mathbb{R} \to \mathbb{R}$ defined by $f(x)=\frac{2x+1}{3}$.
\item $g : \mathcal{P}(\mathbb{N}) \to \mathcal{P}(\mathbb{N})$ defined by $g(X) = \mathbb{N} \setminus X$.
\item $h : \mathbb{N} \times \mathbb{N} \to \mathbb{N}$ defined by $h(m,n) = 2^m(2n+1)-1$ for all $m,n \in \mathbb{N}$.
\end{enumerate}
\begin{backhint}
\hintref{exFindTwoSidedInverses}
For part (c), don't try to write a formula for the inverse of $h$; instead, use the fundamental theorem of arithmetic.
\end{backhint}
\end{exercise}

In light of the correspondences between injections and left inverses, and surjections and right inverses, it may be unsurprising that there is a correspondence between \textit{bijections} and \textit{two-sided inverses}.

\begin{exercise}
\label{exBijectiveIffHasInverse}
Let $f : X \to Y$ be a function. Then $f$ is bijective if and only if $f$ has an inverse.
\end{exercise}



\begin{exercise}
\label{exInverseBijection}
Let $f : X \to Y$ be a bijection. Prove that $f^{-1} : Y \to X$ is a bijection.
\begin{backhint}
\hintref{exInverseBijection}
Use \Cref{exBijectiveIffHasInverse}.
\end{backhint}
\end{exercise}

\begin{exercise}
\label{exCompositeBijection}
Let $f : X \to Y$ and $g : Y \to Z$ be bijections. Prove that $g \circ f : X \to Z$ is a bijection, and write an expression for its inverse in terms of $f^{-1}$ and $g^{-1}$.
\end{exercise}

\begin{exercise}
Let $f : X \to A$ and $g : Y \to B$ be bijections. Prove that there is a bijection $X \times Y \to A \times B$, and describe its inverse.
\hintlabel{exCartesianProductOfBijections}{%
Define $h : X \times Y \to A \times B$ by $h(x,y) = (f(x), g(y))$ for all $x \in X$ and all $y \in Y$; find an inverse for $h$ in terms of the inverses of $f$ and $g$.
}
\end{exercise}

