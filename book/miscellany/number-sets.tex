% !TeX root = ../../infdesc.tex
\section{Constructions of the number sets}
\secbegin{secConstructions}


\begin{exercise}
Write out the elements of $3_{\mathsf{vN}}$ ($=\varnothing^{+++}$) and of $4_{\mathsf{vN}}$.
\end{exercise}

\begin{exercise}
\label{exSizeOfVonNeumannOrdinals}
Recall the definition of von Neumann natural numbers from \Cref{defVonNeumannNaturalNumbers}. Prove that $|n_{\mathsf{vN}}| = n$ for all $n \in \mathbb{N}$.
\end{exercise}


\begin{exercise}
Prove that the recursive definition of $\le$ is equivalent to the definition of `$m \le n$' as `$\exists k \in \mathbb{N},\, m+k=n$'.
\hintlabel{exTwoDefinitionsOfLeqOnN}{%
Define two relations $\le$ and $\le'$ on $\mathbb{N}$, one using the recursive definition and the other using the definition as a logical formula; then use induction to prove that $m \le n \Leftrightarrow m \le' n$ for all $m,n \in \mathbb{N}$.
}
\end{exercise}


\begin{exercise}
\label{exRepresentativesOfElementsOfZ}
Prove that every element of $\mathbb{Z}$, as defined in \Cref{cnsIntegersFromNaturalNumbers}, is equal to exactly one of the following: $[(0,0)]_{\sim}$, or $[(0,n)]_{\sim}$ for some $n>0$, or $[(n,0)]_{\sim}$ for some $n>0$.
\end{exercise}



\begin{exercise}
Prove that the function $i : \mathbb{Z} \to \mathbb{Q}$ defined by $i(a) = \dfrac{a}{1}$ is an injection, and for all $a,b \in \mathbb{Z}$ we have $i(a+b) = i(a)+i(b)$, $i(ab) = i(a)i(b)$, $i(-a) = -i(a)$ and $i(a-b) = i(a)-i(b)$.
\end{exercise}



\begin{exercise}
Let $n > 0$ be composite. Prove that $\mathbb{Z}/n\mathbb{Z}$ is not a field, where zero, unit, addition and multiplication are defined as in \Cref{exZpZIsField}.
\end{exercise}



\begin{exercise}
\label{exInverseIsInvolution}
Let $(X, 0, 1, +, {\cdot})$ be a field. Prove that $-(-x)=x$ for all $x \in X$, and that $(x^{-1})^{-1} = x$ for all nonzero $x \in X$.
\begin{backhint}
\hintref{exInverseIsInvolution}
Prove that $x$ is an additive inverse for $-x$ (in the sense of \Cref{axField}(F4)) and use uniqueness of additive inverses. Likewise for $x^{-1}$.
\end{backhint}
\end{exercise}


\begin{exercise}
\label{exMinusOneSquaredIsOne}
Let $(X,0,1,+,{\cdot})$ be a field. Prove that $(-1) \cdot x = -x$ for all $x \in X$, and that $(-x)^{-1} = -(x^{-1})$ for all nonzero $x \in X$.
\end{exercise}



\begin{exercise}
Let $(X,0,1,+,{\cdot})$ be a field. Prove that if $X$ is finite, then there is no relation $\le$ on $X$ such that $(X,0,1,+,{\cdot},{\le})$ is an ordered field.
\end{exercise}
