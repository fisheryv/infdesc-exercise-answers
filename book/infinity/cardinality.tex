% !TeX root = ../../infdesc.tex
\section{Cardinality}
\secbegin{secCardinality}


\begin{exercise}
\label{exIntervalFromMinusOneToOneHasCardinalityOfContinuum}
Consider the function $f : \mathbb{R} \to (-1,1)$ defined by
\[ f(x) = \dfrac{x}{1+|x|} \]
for all $x \in \mathbb{R}$. Prove that $f$ is a bijection, and deduce that $|(-1,1)| = \mathfrak{c}$.
\end{exercise}

\begin{exercise}
\label{exRealsHaveSameCardinalityAsIntervals}
Let $a,b \in \mathbb{R}$ with $a<b$. Prove that each of the sets $(a,b)$, $[a,b)$, $(a,b]$ and $[a,b]$ has cardinality $\mathfrak{c}$.
\hintlabel{exAllOpenInvervalsHaveCardinalityOfContinuum}{%
Start by finding a bijection $(-1,1) \to (a,b)$ and apply \Cref{exIntervalFromMinusOneToOneHasCardinalityOfContinuum}.
}
\end{exercise}


\begin{exercise}
Prove that $n < \aleph_0$ for all $n \in \mathbb{N}$.
\end{exercise}

\begin{exercise}
Prove that $\le$ is a reflexive, transitive relation on $\mathsf{Card}$. That is, prove that:
\begin{enumerate}[(a)]
\item $\kappa \le \kappa$ for all cardinal numbers $\kappa$; and
\item For all cardinal numbers $\kappa, \lambda, \mu$, if $\kappa \le \lambda$ and $\lambda \le \mu$, then $\kappa \le \mu$.
\end{enumerate}
\end{exercise}


\begin{exercise}
Prove that $\sim$ is an equivalence relation on $[\kappa]$.
\end{exercise}

% We prove that $\sim$ is an equivalence relation:
% \begin{itemize}
% \item (\textbf{Reflexivity}) For all $a \in [\kappa]$, we have $a = (g \circ f)^0(a)$, and so $a \sim a$.
% \item (\textbf{Symmetry}) Let $a, b \in [\kappa]$ and suppose that $a \sim b$. Then for some $n \in \mathbb{N}$ we have either $(g \circ f)^n(a) = b$ or $(g \circ f)^n(b) = a$. But this is also what it means for $b \sim a$ to be true!
% \item (\textbf{Transitivity}) Let $a,b,c \in [\kappa]$ and suppose that $a \sim b$ and $b \sim c$. Then there exist $m,n \in \mathbb{N}$ such that one of the following cases holds:
% \begin{itemize}
% \item $(g \circ f)^m(a) = b$ and $(g \circ f)^n(b) = c$. In this case, we have
% \[ (g \circ f)^{m+n}(a) ~=~ (g \circ f)^n( (g \circ f)^m (a) ) ~=~ (g \circ f)^n(b) ~=~ c \]
% and so $a \sim c$.
% \item $(g \circ f)^m(a) = b$ and $(g \circ f)^n(c) = b$. Then $(g \circ f)^m(a) = (g \circ f)^n(c)$. Since $g \circ f$ is an injection, we may cancel it from both sides of the equation as many times as we can, meaning that either
% \[ (g \circ f)^{m-n}(a) = c \quad \text{or} \quad (g \circ f)^{n-m}(c) = a \]
% But both of these imply that $a \sim c$.
% \item $(g \circ f)^m(b) = a$ and $(g \circ f)^n(b) = c$. In this case we either have
% \[ (g \circ f)^{n-m}(a) = (g \circ f)^n(b) = c \quad \text{or} \quad (g \circ f)^{m-n}(c) = (g \circ f)^m(b) = a \]
% But both of these imply that $a \sim c$.
% \item $(g \circ f)^m(b)=a$ and $(g \circ f)^n(c) = b$. In this case, we have
% \[ (g \circ f)^{m+n}(c) = (g \circ f)^m( (g \circ f)^n(c) ) = (g \circ f)^m(b) = a \]
% and so $a \sim c$.
% \end{itemize}
% In all four cases, we have $a \sim c$, as required.
% \end{itemize}



\begin{exercise}
Prove that $p$ and $q$ are well-defined, and that $q$ is an inverse for $p$.
\end{exercise}


\begin{exercise}
Prove that each of the functions defined in the above three cases is a bijection.
\end{exercise}


\begin{exercise}
Let $\mathcal{F}(\mathbb{N})$ be the set of all finite subsets of $\mathbb{N}$, and define $f : \mathcal{F}(\mathbb{N}) \to \mathbb{N}$ by
\[ f(U) ~=~ \sum_{a \in U} 10^a ~=~ \sum_{k = 1}^n 10^{a_k} \]
for all $U = \{ a_1, a_2, \dots, a_n \} \in \mathcal{F}(\mathbb{N})$. Put another way, $f(U)$ is the natural number whose decimal expansion has a $1$ in the $r^{\text{th}}$ position (counting from $r=0$) if $r \in U$, and a $0$ otherwise. For example
\[ f(\{ 0, 1, 3, 4, 8 \}) = 100011011 \quad \text{and} \quad f(\varnothing) = 0 \]
Prove that $f$ is injective, and use the Cantor--Schr\"{o}der--Bernstein theorem to deduce that $\mathcal{F}(\mathbb{N})$ is countably infinite.
\end{exercise}

\begin{exercise}
Let $X$, $Y$ and $Z$ be sets. Prove that if $|X| = |Z|$ and $X \subseteq Y \subseteq Z$, then $|X| = |Y| = |Z|$.
\end{exercise}