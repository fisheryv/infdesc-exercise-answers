% !TeX root = ../../infdesc.tex
\section{Cardinal arithmetic}
\secbegin{secCardinalArithmetic}


\begin{exercise}
Prove parts (b) and (c) of \Cref{thmPropertiesOfCardinalAddition}.
\end{exercise}


\begin{exercise}
Prove that $\mathfrak{c} \cdot \mathfrak{c} = \mathfrak{c}$.
\hintlabel{exCTimesCEqualsC}{%
A nice trick is to construct an injection $[0,1) \times [0,1) \to [0,1)$ using decimal expansions, and then invoke the Cantor--Schr\"{o}der--Bernstein theorem.
}
\end{exercise}

\begin{exercise}
\label{exCardinalityOfQuotient}
Let $X$ be a set and let $\sim$ be an equivalence relation on $X$. Prove that if $\kappa$ is a cardinal number such that $|[a]_{\sim}| = \kappa$ for all $a \in X$, then $|X| = |X/{\sim}| \cdot \kappa$.
\end{exercise}

\begin{exercise}
Prove parts (b), (c) and (d) of \Cref{thmPropertiesOfCardinalMultiplication}.
\end{exercise}



\begin{exercise}
\label{exCardinalExponentiationIsMonotone}
Prove that for all cardinal numbers $\kappa, \lambda, \mu$, if $\lambda \le \mu$, then $\lambda^{\kappa} \le \mu^{\kappa}$.
\end{exercise}


\begin{exercise}
Prove parts (b) and (c) of \Cref{thmPropertiesOfCardinalExponentiation}.
\end{exercise}


\begin{exercise}
Prove that $\mathfrak{c}^{\mathfrak{c}} = 2^{\mathfrak{c}}$.
\end{exercise}

\begin{exercise}
Prove that $\aleph_0^{\aleph_0^{\aleph_0}} = 2^{\mathfrak{c}}$.
\end{exercise}


\begin{exercise}
\label{exFunctionFromDisjointUnionToUnionIsBijectionIfPairwiseDisjoint}
Let $\{ X_i \mid i \in I \}$ be a family of sets indexed by a set $I$, and define a function
\[ q : \bigsqcup_{i \in I} X_i \to \bigcup_{i \in I} X_i \]
by $q(i,a) = a$ for all $i \in I$ and $a \in X_i$. Prove that if the sets $X_i$ for $i \in I$ are pairwise disjoint, then $q$ is a bijection.
\end{exercise}

\begin{exercise}
Let $(a_n)_{n \in \mathbb{N}}$ be a sequence of natural numbers, and let $I = \{ n \in \mathbb{N} \mid a_n > 0 \}$. Prove that
\[ \sum_{n \in \mathbb{N}} a_n = \begin{cases} \aleph_0 & \text{if $I$ is infinite} \\ \sum_{k=1}^n a_{n_k} & \text{if $I = \{ n_k \mid k \in [n] \}$ is finite} \end{cases} \]
\end{exercise}


\begin{exercise}
Prove that $1 \times 2 \times 3 \times 4 \times \cdots = 2^{\aleph_0}$.
\end{exercise}

\begin{exercise}
Prove that there do not exist cardinal numbers $\{ \kappa_n \mid n \in \mathbb{N} \}$ such that $\kappa_n \ne 1$ for all $n \in \mathbb{N}$ and $\displaystyle \prod_{n \in \mathbb{N}} \kappa_n = \aleph_0$.
\end{exercise}