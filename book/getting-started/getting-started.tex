% !TeX root = ../../infdesc.tex
\hintsection{\Cref*{chGettingStarted}}

\begin{exercise}
Think of an example of a true proposition, a false proposition, a proposition whose truth value you don't know, and a statement that is not a proposition.
\end{exercise}

\begin{solution}
The following are propositions as required:
\begin{itemize}
\item \textbf{True proposition}: Abraham Lincoln was the 16th President of the United States;
\item \textbf{False proposition}: The sum of the interior angles of a triangle on the plane is greater than 180$\degree$;
\item \textbf{Unknown proposition}: The earth will be destroyed in 2299;
\item \textbf{Not a proposition}: Happy birthday to you.
\end{itemize}
\end{solution}

%%%%%%%%%%%%%%%%%%%%%%%%%%%%%%%%%%%%%%%%%%%%%%%%%%%%%%%%%%%

\begin{exercise}
Find the binary, ternary, octal, decimal, hexadecimal and base-$36$ expansions of the number $21127$, using the letters $\mathrm{A}$--$\mathrm{F}$ as additional digits for the hexadecimal expansion (representing the numbers $10$--$15$, respectively), and the letters $\mathrm{A}$--$\mathrm{Z}$ as additional digits for the base-$36$ expansion.
\end{exercise}

\begin{solution}
\begin{align*}
21127 &= 101001010000111_{(2)} \\
&= 1001222111_{(3)} \\
&= 51207_{(8)}\\
&= 21127_{(10)}\\
&= 5287_{(16)}\\
&= \mathrm{KK}7_{(36)}
\end{align*}
\end{solution}

%%%%%%%%%%%%%%%%%%%%%%%%%%%%%%%%%%%%%%%%%%%%%%%%%%%%%%%%%%%

\begin{exercise}
\label{exOneDividesEveryIntegerDividesZero}
Prove that $1$ divides every integer, and that every integer divides $0$.
\end{exercise}

\begin{solution}
According to the definition of divisibility, we say $a$ divides $b$, if $b=qa$. We know $b=b\times1$, so that $1$ divides every integer; 

For all $a \in \mathbb{Z}$, we have $0=0\times a$, so that every integer divides $0$. 
\end{solution}

%%%%%%%%%%%%%%%%%%%%%%%%%%%%%%%%%%%%%%%%%%%%%%%%%%%%%%%%%%%

\begin{exercise}
\label{exDivisibilityIsLinear}
Let $a,b,d \in \mathbb{Z}$. Suppose that $d$ divides $a$ and $d$ divides $b$. Given integers $u$ and $v$, prove that $d$ divides $au+bv$.
\end{exercise}

\begin{solution}
According to the definition of divisibility, 
\begin{align*}
&\because d \text{ divides } a\\
&\therefore a=pd \quad(p \in \mathbb{Z})\\
&\because d \,\text{ divides }\, b\\
&\therefore b=qd \quad(q \in \mathbb{Z})\\
&\therefore au+bv = upd + vqd = (up+vq)d\\ 
\end{align*}
so that $d$ divides $au+bv$.
\end{solution}

%%%%%%%%%%%%%%%%%%%%%%%%%%%%%%%%%%%%%%%%%%%%%%%%%%%%%%%%%%%

\begin{exercise}
Prove that if an integer $a$ leaves a remainder of $r$ when divided by an integer $b$, then $a$ leaves a remainder of $r$ when divided by $-b$.
\end{exercise}

\begin{solution}
According to the division theorem, an integer $a$ divided by an integer $b$ leaves a remainder of $r$,so there is some $q \in \mathbb{Z}$ such that \\
\[a = qb+r \quad 0 \le r < b \tag{1}.\] 

Suppose $a$ divided by $-b$ leaves $m$, so there is some $p \in \mathbb{Z}$ such that \\
\[a=p(-b)+m \quad 0 \le m < b \tag{2}.\]

$(1)-(2)=qb+r+pb-m=0$, rearranged formula is
\[(p+q)b=m-r\]
\begin{align*}
&\because 0 \le m < b \text{ and } 0 \le r < b \\
&\therefore -b < (p+q)b=m-r < b \\
&\therefore -1 < p+q < 1 \\
&\because (p+q) \in \mathbb{Z} \\
&\therefore p+q=0 \\
&\therefore (p+q)b=m-r=0 \\
&\therefore m=r \\
\end{align*}
so $a$ leaves a remainder of $r$ when divided by $-b$.
\end{solution}

%%%%%%%%%%%%%%%%%%%%%%%%%%%%%%%%%%%%%%%%%%%%%%%%%%%%%%%%%%%

\begin{exercise}
Find the base-$17$ expansion of $408\,735\,787$ and the base-$36$ expansion of $1\,442\,151\,747$.
\end{exercise}

\begin{solution}
\[408\,735\,787 = GFEDCBA_{(17)}\]
\[1\,442\,151\,747 = 2DV152_{(36)}\]
\end{solution}

%%%%%%%%%%%%%%%%%%%%%%%%%%%%%%%%%%%%%%%%%%%%%%%%%%%%%%%%%%%

\begin{exercise}
Interpret the operations of subtraction and division as geometric transformations of the real number line.
\end{exercise}

\begin{solution}
Interpret \textbf{subtraction} as geometric transformations of the real number line.

If we consider addition as moving to right, subtraction can be considered as moving to left. Concretely, take two copies of the number line, one above the other, with the same choice of unit length; move the $b$ of the lower number line beneath the point $a$ of the upper number line. Then $a-b$ is the point on the upper number line lying above the point $0$ of the lower number line. For example, $5-3=2$, here is an illustration of the fact that $5-3=2$:
\begin{center}
\fitwidthc{0.9}{\begin{tikzpicture}
\draw[latex-latex] (-0.5,0) -- (10.5,0) ; 
\foreach \x in  {0,1,2,3,4,5,6,7,8,9,10}
\draw[shift={(\x,0)}] (0pt,3pt) -- (0pt,-3pt);
\foreach \x in {0,1,2,3,4,5,6,7,8,9,10}
\draw[shift={(\x,0)}] (0pt,0pt) -- (0pt,3pt) node[above] {$\x$};
\draw[*-*] (1.92,0) -- (5.08,0);
\draw[very thick] (2,0) -- (5,0);

\draw[latex-latex] (-0.5,-1) -- (10.5,-1) ; 
\foreach \x in  {-2,-1,0,1,2,3,4,5,6,7,8}
\draw[shift={(\x+2,-1)},color=black] (0pt,3pt) -- (0pt,-3pt);
\foreach \x in {-2,-1,0,1,2,3,4,5,6,7,8}
\draw[shift={(\x+2,-1)},color=black] (0pt,0pt) -- (0pt,-3pt) node[below] {$\x$};
\draw[->] (2,-0.8) -- (2,-0.2) ;
\draw[dashed] (5,-1) -- (5,0) ;
\end{tikzpicture}}
\end{center}

Interpret \textbf{division} as geometric transformations of the real number line.

If we consider multiplication as zoom in, division can be considered as zoom out. Concretely, take two copies of the number line, one above the other; align the $0$ points on both number lines, and compress the lower number line evenly until the point $b$ on the lower number line is below the point $1$ on the upper number line, Then $a/b$ is the point on the upper number line lying above $a$ on the lower number line. For example, $20/5=4$, here is an illustration of the fact that $20/5=4$:
\begin{center}
\fitwidthc{0.9}{\begin{tikzpicture}
\draw[latex-latex] (-5.5,0) -- (8.5,0) ; 
\foreach \x in  {0,1,2,3,4}
\draw[shift={(2.5*\x-4,0)},color=black] (0pt,3pt) -- (0pt,-3pt);
\foreach \x in {0,1,2,3,4}
\draw[shift={(2.5*\x-4,0.2)},color=black] node[above] {$\x$};
\draw[*-*] (-1.58,0) -- (6.08,0);
\draw[very thick] (-1.5,0) -- (6,0);

\draw[latex-latex] (-5.5,-1) -- (8.5,-1) ; 
\foreach \x in  {-2,-1,0,1,2,3,4,5,6,7,8,9,10,11,12,13,14,15,16,17,18,19,20,21,22,23,24}
\draw[shift={(0.5*\x-4,-1)}] (0pt,3pt) -- (0pt,-3pt);
\foreach \x in {-2,-1,0,1,2,3,4,5,6,7,8,9,10,11,12,13,14,15,16,17,18,19,20,21,22,23,24}
\draw[shift={(0.5*\x-4,-1.2)}] node[below] {$\text{\footnotesize\x}$};
\draw[dashed] (-4,-1) -- (-4,0) ;
\draw[dashed] (-1.5,-1) -- (-1.5,0) ;
\draw[->] (6,-0.8) -- (6,-0.2) ;
\end{tikzpicture}}
\end{center}
\end{solution}

%%%%%%%%%%%%%%%%%%%%%%%%%%%%%%%%%%%%%%%%%%%%%%%%%%%%%%%%%%%

\begin{exercise}
Let $r$ be a rational number and let $a$ be an irrational number. Prove that it is possible that $ra$ be rational, and it is possible that $ra$ be irrational.
\end{exercise}

\begin{solution}
We can prove it by example,

$r$ is a rational number and $a$ is an irrational number, let $r = 0$, so that $ra=0$ is a rational number. 

Again let $r = 1$, so that $ra=a$ is an irrational number.
\end{solution}

%%%%%%%%%%%%%%%%%%%%%%%%%%%%%%%%%%%%%%%%%%%%%%%%%%%%%%%%%%%

\begin{exercise}
Write down a polynomial of degree $4$ over $\mathbb{R}$ which is not a polynomial over $\mathbb{Q}$.
\end{exercise}

\begin{solution}
\[p(x) = ax^4+bx^3+cx^2+dx+e \quad x \in \mathbb{R} \setminus \mathbb{Q}\]
\end{solution}

%%%%%%%%%%%%%%%%%%%%%%%%%%%%%%%%%%%%%%%%%%%%%%%%%%%%%%%%%%%

\begin{exercise}
\label{exComplexNumberAsRootOfQuadraticOverR}
Let $\alpha = a+bi$ be a complex number, where $a,b \in \mathbb{R}$. Prove that $\alpha$ is the root of a quadratic polynomial over $\mathbb{R}$, and find the other root of this polynomial.
\end{exercise}

\begin{solution}
According to the quadratic formula, a quadratic polynomial $p(x) = x^2+mx+n$ has 2 roots over $\mathbb{C}$, which are
\[ \frac{-m+\sqrt{m^2-4n}}{2} \quad \text{and} \quad \frac{-m-\sqrt{m^2-4n}}{2} \]
rearrange the root formula, that
\[ \frac{-m}{2}+\frac{\sqrt{4n-m^2}}{2}i \quad \text{and} \quad \frac{-m}{2}-\frac{\sqrt{4n-m^2}}{2}i \]
Because $m,n$ are coefficients, let $\frac{-m}{2} = a$ and $\frac{\sqrt{4n-m^2}}{2} = b$, then we have $a+bi$ and $a-bi$ as two roots of a quadratic polynomial.

Now, let’s construct a polynomial $p(x)$ such that $P(\alpha) = 0$. We can multiply the factors $(x - (a+bi))$ and $(x - (a-bi))$ to obtain:
\[p(x) = (x - (a + bi))(x - (a - bi)).\]
Expanding the product, we get:
\[p(x) = (x - a - bi)(x - a + bi) = ((x - a) - bi)((x - a) + bi) = (x - a)^2 - (bi)^2=(x-a)^2+b^2.\]
Since $a,b \in \mathbb{R}$, $(x - a)^2$ and $b^2$ are also real numbers. Thus, the coefficients of $p(x)$ are real numbers, which means $p(x)$ is a quadratic polynomial over $\mathbb{R}$. The other root is $a-bi$.
\end{solution}

%%%%%%%%%%%%%%%%%%%%%%%%%%%%%%%%%%%%%%%%%%%%%%%%%%%%%%%%%%%

\begin{exercise}
\label{exDiscriminantRealRoots}
\index{discriminant}
Let $a,b \in \mathbb{C}$ and let $p(x)=x^2+ax+b$. The value $\Delta=a^2-4b$ is called the \textbf{discriminant} of $p$. Prove that $p$ has two roots if $\Delta \ne 0$ and one root if $\Delta = 0$. Moreover, if $a,b \in \mathbb{R}$, prove that $p$ has no real roots if $\Delta < 0$, one real root if $\Delta = 0$, and two real roots if $\Delta > 0$.
\end{exercise}

\begin{solution}
According to the quadratic formula, a quadratic polynomial $p(x) = x^2+ax+b$ has 2 roots over $\mathbb{C}$, which are
\[ \frac{-a+\sqrt{a^2-4b}}{2} \quad \text{and} \quad \frac{-a-\sqrt{a^2-4b}}{2} \]
\begin{itemize}
\item if $\Delta=a^2-4b=0$, then $ \frac{-a+\sqrt{a^2-4b}}{2} = \frac{-a-\sqrt{a^2-4b}}{2}$, so that there is one root.
\item if $\Delta=a^2-4b \ne 0$, then $ \frac{-a+\sqrt{a^2-4b}}{2} \ne \frac{-a-\sqrt{a^2-4b}}{2}$, so that there are two roots.
\end{itemize}
if $a,b \in \mathbb{R}$ and $\Delta < 0$, $\sqrt{a^2-4b}$ does not be defined in $\mathbb{R}$, so that there is no real root.
\end{solution}