% !TeX root = ../../infdesc.tex
\hintsection{\Cref*{chGettingStarted}}

\begin{exercise}
Think of an example of a true proposition, a false proposition, a proposition whose truth value you don't know, and a statement that is not a proposition.
\end{exercise}

\begin{solution}
The following are propositions as required:
\begin{itemize}
\item \textbf{True proposition}: Abraham Lincoln was the 16th President of the United States;
\item \textbf{False proposition}: The sum of the interior angles of a triangle on the plane is greater than 180$\degree$;
\item \textbf{Unknown proposition}: The earth will be destroyed in 2299;
\item \textbf{Not a proposition}: Happy birthday to you.
\end{itemize}
\end{solution}

%%%%%%%%%%%%%%%%%%%%%%%%%%%%%%%%%%%%%%%%%%%%%%%%%%%%%%%%%%%

\begin{exercise}
Find the binary, ternary, octal, decimal, hexadecimal and base-$36$ expansions of the number $21127$, using the letters $\mathrm{A}$--$\mathrm{F}$ as additional digits for the hexadecimal expansion (representing the numbers $10$--$15$, respectively), and the letters $\mathrm{A}$--$\mathrm{Z}$ as additional digits for the base-$36$ expansion.
\end{exercise}

\begin{solution}
\begin{align*}
21127 &= 101001010000111_{(2)} \\
&= 1001222111_{(3)} \\
&= 51207_{(8)}\\
&= 21127_{(10)}\\
&= 5287_{(16)}\\
&= \mathrm{KK}7_{(36)}
\end{align*}
\end{solution}

%%%%%%%%%%%%%%%%%%%%%%%%%%%%%%%%%%%%%%%%%%%%%%%%%%%%%%%%%%%

\begin{exercise}
\label{exOneDividesEveryIntegerDividesZero}
Prove that $1$ divides every integer, and that every integer divides $0$.
\end{exercise}

\begin{solution}
According to the definition of divisibility, we say $a$ divides $b$, if $b=qa$. We know $b=b\times1$, so that $1$ divides every integer; 

For all $a \in \mathbb{Z}$, we have $0=0\times a$, so that every integer divides $0$. 
\end{solution}

%%%%%%%%%%%%%%%%%%%%%%%%%%%%%%%%%%%%%%%%%%%%%%%%%%%%%%%%%%%

\begin{exercise}
\label{exDivisibilityIsLinear}
Let $a,b,d \in \mathbb{Z}$. Suppose that $d$ divides $a$ and $d$ divides $b$. Given integers $u$ and $v$, prove that $d$ divides $au+bv$.
\end{exercise}

\begin{solution}
According to the definition of divisibility, 
\begin{align*}
&\because d \text{ divides } a\\
&\therefore a=pd \quad(p \in \mathbb{Z})\\
&\because d \,\text{ divides }\, b\\
&\therefore b=qd \quad(q \in \mathbb{Z})\\
&\therefore au+bv = upd + vqd = (up+vq)d\\ 
\end{align*}
so that $d$ divides $au+bv$.
\end{solution}

%%%%%%%%%%%%%%%%%%%%%%%%%%%%%%%%%%%%%%%%%%%%%%%%%%%%%%%%%%%

\begin{exercise}
Prove that if an integer $a$ leaves a remainder of $r$ when divided by an integer $b$, then $a$ leaves a remainder of $r$ when divided by $-b$.
\end{exercise}

%%%%%%%%%%%%%%%%%%%%%%%%%%%%%%%%%%%%%%%%%%%%%%%%%%%%%%%%%%%

\begin{exercise}
Find the base-$17$ expansion of $408\,735\,787$ and the base-$36$ expansion of $1\,442\,151\,747$.
\end{exercise}

\begin{solution}

\[408\,735\,787 = GFEDCBA_{(17)}\]

\[1\,442\,151\,747 = 2DV152_{(36)}\]
\end{solution}


\begin{exercise}
Interpret the operations of subtraction and division as geometric transformations of the real number line.
\end{exercise}


\begin{exercise}
Let $r$ be a rational number and let $a$ be an irrational number. Prove that it is possible that $ra$ be rational, and it is possible that $ra$ be irrational.
\end{exercise}

\begin{solution}
We can prove it by example,

$r$ is a rational number and $a$ is an irrational number, let $r = 0$, so that $ra=0$ is a rational number. 

Again let $r = 1$, so that $ra=a$ is an irrational number.
\end{solution}


\begin{exercise}
Write down a polynomial of degree $4$ over $\mathbb{R}$ which is not a polynomial over $\mathbb{Q}$.
\end{exercise}


\begin{exercise}
\label{exComplexNumberAsRootOfQuadraticOverR}
Let $\alpha = a+bi$ be a complex number, where $a,b \in \mathbb{R}$. Prove that $\alpha$ is the root of a quadratic polynomial over $\mathbb{R}$, and find the other root of this polynomial.
\end{exercise}


\begin{exercise}
\label{exDiscriminantRealRoots}
\index{discriminant}
Let $a,b \in \mathbb{C}$ and let $p(x)=x^2+ax+b$. The value $\Delta=a^2-4b$ is called the \textbf{discriminant} of $p$. Prove that $p$ has two roots if $\Delta \ne 0$ and one root if $\Delta = 0$. Moreover, if $a,b \in \mathbb{R}$, prove that $p$ has no real roots if $\Delta < 0$, one real root if $\Delta = 0$, and two real roots if $\Delta > 0$.
\end{exercise}
